%--------------------------INITIALISATION DU DOCUMENT---------------------%
%											%														%											%
%                        >>>> Ne pas modifier cette partie <<<<			%
																																					
\documentclass[letterpaper,12pt,oneside,final]{book}



%%
%%  Version: 2014-10-28
%%
%%  Accepte les caractères accentués dans le document (UTF-8).
\usepackage[utf8]{inputenc}
%%
%% Support pour l'anglais et le français (français par défaut).
%\usepackage[cyr]{aeguill}
\usepackage{lmodern}      % Police de caractères plus complète et généralement indistinguable visuellement de la police standard de LaTeX (Computer Modern).
\usepackage[T1]{fontenc}  % Bon encodage des caractères pour qu'Acrobat Reader reconnaisse les accents et les ligatures telles que ffi.
\usepackage[english,frenchb]{babel} % le langage par défaut est le dernier de la liste, c'est-à-dire français
%%
%% Charge le module d'affichage graphique.
\usepackage{graphicx}
\usepackage{epstopdf}  % Permet d'utiliser des .eps avec pdfLaTeX.
%%
%% Recherche des images dans les répertoires.
\graphicspath{{./images/}{./dia/}{./gnuplot/}}
%%
%% Un float peut apparaître seulement après sa définition, jamais avant.
\usepackage{flafter,placeins}
%%
%% Utilisation de natbib pour les citations et la bibliographie.
\usepackage{natbib}
%%
%% Autres packages.
\usepackage{amsmath,color,soulutf8,longtable,colortbl,setspace,ifthen,xspace,url,pdflscape,tikz,pgfplots}
%%
%% Support des acronymes.
\usepackage[nolist]{acronym}
\onehalfspacing                % Interligne 1.5.
%%
%% Définition d'un style de page avec seulement le numéro de page à
%% droite. On s'assure aussi que le style de page par défaut soit
%% d'afficher le numéro de page en haut à droite.
\usepackage{fancyhdr}
\fancypagestyle{pagenumber}{\fancyhf{}\fancyhead[R]{\thepage}}
\renewcommand\headrulewidth{0pt}
\makeatletter
\let\ps@plain=\ps@pagenumber
\makeatother
%%
%% Module qui permet la création des bookmarks dans un fichier PDF.
%\usepackage[dvipdfm]{hyperref}
\usepackage{hyperref}
\usepackage{caption}  % Hyperlien vers la figure plutôt que son titre.

\usepackage{esint}
\usepackage{geometry}
\usepackage{enumerate}




\begin{document}
\setlength{\parindent}{0pt}
\newcommand{\norme}[1]{\left\Vert #1\right\Vert}
%--------------------------------------------------------------------------------------%

%--------------------------PAGE DE COUVERTURE------------------------------%

% A REMPLIR PAR L'ETUDIANT: 

\newcommand\monPrenom{William}		%PRENOM
\newcommand\monNom{Trépanier}			%NOM
\newcommand\monMatricule{1952594}	%MATRICULE
\newcommand\monGroupe{1}		%GROUPE

%------------------------ Ne pas modifier la ligne suivante --------------%
%\newgeometry{tmargin=2.0cm, bmargin=2.0cm, lmargin=2.25cm, rmargin=2.25cm, headsep=1.0cm}
\newgeometry{top=2cm}
\definecolor{gris1}{gray}{0.75}

\newcommand{\encadre}[1]{
\setlength\fboxsep{5mm}\setlength\fboxrule{1pt}
\begin{center}
\fcolorbox{black}{gris1}{
\begin{minipage}{0.94\textwidth}{#1}\end{minipage}}
\end{center}}

% encadre blanc
\newcommand{\boite}[1]{
\setlength\fboxsep{5mm}\setlength\fboxrule{1pt}
\begin{center}
\fcolorbox{black}{white}{
\begin{minipage}{0.5\textwidth}{#1}\end{minipage}}
\end{center}}


%\begin{document}

\thispagestyle{empty}

{
\centering

\encadre{
\begin{center}
\bf
{\Large \scshape 
Polytechnique Montr\'eal
\\
D\'epartement de Math\'ematiques et de G\'enie Industriel
}
\\
{\Huge
\

MTH1102 - Calcul II
\\
\'Et\'e 2019 - Trimestre court

\

Devoir 4

}
\end{center}
}

\vfill

\fcolorbox{black}{white}{
\begin{minipage}{0.94\linewidth}

\vspace{5mm}

{\bf \Large Nom: }\monNom \hspace{20mm} {\bf \Large Pr\'enom: }\monPrenom%\rule[-1mm]{56mm}{0.6pt}

\vspace{8mm}

{\bf \Large Matricule: }\monMatricule \hspace{20mm} {\bf \Large Section: }\monGroupe

%\vspace{8mm}
%
%{\bf \Large Signature: }\rule[-1mm]{126mm}{0.6pt}
%
%
%\vspace{5mm}

\end{minipage}}

\vfill


{
\renewcommand{\arraystretch}{1.5}
\begin{center}
\begin{tabular}{|c|c|c||c|} \hline
{\bf \Large Question}							& {\bf \Large Autres}			& 	{\bf \Large Bonus}	&  \\ 
{\bf \Large corrig\'ee}						& {\bf \Large  questions}	&		{\bf \Large \LaTeX}	& {\bf \Large Total} \\ \hline
\hspace{20mm}			{\Huge \strut}	& \hspace{20mm}						&		\hspace{20mm}				&\hspace{20mm} \\
\hspace{20mm}			{\Huge \strut}	& \hspace{20mm}						&		\hspace{20mm} 			& \hspace{20mm} {\Large /10} \\ \hline
\end{tabular}
\end{center}
}


\vfill

}

\restoregeometry
%\end{document}
%-------------------------------------------------------------------------%


%========================= Début des réponses ============================%


%-----------------------------QUESTION 1 ----------------------------------%
\section*{Question 1}

Le paraboloïde $z_1=x^2+y^2+2$ se retrouve au dessus du paraboloïde 
    $z_2=2x^2+2y^2$. En passant aux coordonnées cylindriques, on obtient :
    $z_1 = r^2 + 2$ et $z_2 = 2r^2$. Pour trouver les bornes de $r$ sur sa 
    projection dans le plan $xy$, il faut trouver l'intersection des deux
    paraboloïdes :

\[ 2r^2 = r^2 + 2 \Leftrightarrow r^2-2 = 0 \Leftrightarrow 
    (r+\sqrt{2})(r-\sqrt{2}) = 0 \Rightarrow r = \sqrt{2},\; r \ge 0 \]

Ainsi, nous obtenons :

\[ E=\{(r,\theta, z)\;|\;0\le r\le \sqrt{2},\;0\le \theta \le 2\pi,\;
    2r^2 \le z \le r^2+2\} \]

Quant à l'intégrale : 

\begin{multline*}
    J_1 = \iiint\limits_E (x^2+y^2)^{3/2}\;dV = \int_0^{2\pi} \int_0^{\sqrt{2}}
        \int_{2r^2}^{r^2+2} (r^2)^{3/2}\cdot r\;dzdrd\theta = \\
    \int_0^{2\pi} \int_0^{\sqrt{2}} \int_{2r^2}^{r^2+2} r^4\;dzdrd\theta =
        \int_0^{2\pi} \int_0^{\sqrt{2}} \big(2r^4 - r^6\big)\;drd\theta =
        \int_0^{2\pi} \Big(2^{5/2}\big(\frac{2}{5}-\frac{2}{7}\big)\Big)\;
        d\theta = \\
    2\pi\cdot\frac{16\sqrt{2}}{35} = \frac{32\sqrt{2}\pi}{35}
\end{multline*}

%-----------------------------QUESTION 2 ----------------------------------%
\newpage\section*{Question 2}

Nous avons deux sphères, soit $S_1 := x^2+y^2+z^2 = 4$ et 
    $S_2 := x^2+y^2+z^2 = 16$ dont nous pouvons transformer en coordonnées
    sphériques : $S_1 := \rho^2 = 2^2$ et $S_2 := \rho^2 = 4^2$. Puisque
    $x \ge 0 \Rightarrow \frac{-\pi}{2} \le \theta \le \frac{\pi}{2}$. Aussi,
    considérant $z \ge 0 \Rightarrow 0 \le \phi \le \frac{\pi}{2}$.
    Nous avons donc la région $E$ suivante : 

\[ E=\Big\{(\rho,\theta, \phi)\;|\;2\le \rho\le 4,\;\frac{-\pi}{2}\le \theta \le 
    \frac{\pi}{2},\; 0 \le \phi \le \frac{\pi}{2}\Big\} \]

Quant à l'intégrale : 

\begin{multline*}
    J_2 = \iiint\limits_E (x+y^2)\;dV = \int_0^{\frac{\pi}{2}}
        \int_{\frac{-\pi}{2}}^{\frac{\pi}{2}} \int_2^4 
        \Big[(\rho\sin\phi\cos\theta + \\ \rho^2sin^2\phi\sin^2\theta)
        \rho^2\sin\phi\Big] \; d\rho\,d\theta\,d\phi = \\ 
    \int_0^{\frac{\pi}{2}} \int_{\frac{-\pi}{2}}^{\frac{\pi}{2}} \int_2^4
        \rho^3 \sin^2\phi(\cos\theta + \rho\sin\phi\sin^2\theta) \;
        d\rho\,d\theta\,d\phi = \\
    \int_0^{\frac{\pi}{2}} \int_{\frac{-\pi}{2}}^{\frac{\pi}{2}} \int_2^4
        \rho^3 \sin^2\phi\cos\theta \; d\rho\,d\theta\,d\phi \; + \; %2e int
        \int_0^{\frac{\pi}{2}} \int_{\frac{-\pi}{2}}^{\frac{\pi}{2}} \int_2^4
        \rho^4\sin^3\phi\sin^2\theta \; d\rho\,d\theta\,d\phi = \\
    60 \int_0^{\frac{\pi}{2}} \int_{\frac{-\pi}{2}}^{\frac{\pi}{2}} 
        \sin^2\phi\cos\theta \; d\theta\,d\phi \; + \; %2e int
        \frac{992}{5} \int_0^{\frac{\pi}{2}} \int_{\frac{-\pi}{2}}^{\frac{\pi}{2}}
        \sin^3\phi\sin^2\theta \; d\theta\,d\phi = \\
    120 \int_0^{\frac{\pi}{2}} \sin^2\phi \; d\phi \; +  \;%2e int
        \frac{992}{5} \int_0^{\frac{\pi}{2}} \int_{\frac{-\pi}{2}}^{\frac{\pi}{2}}
        \sin^3\phi\cdot \frac{1}{2}(1-\cos(2\theta)) \; d\theta\,d\phi = \\
    120 \int_0^{\frac{\pi}{2}} \frac{1}{2}(1-\cos(2\phi)) \; d\phi \;+\;%2e int
        \frac{992\pi}{10} \int_0^{\frac{\pi}{2}} \sin^3\phi\; d\phi = \\
    30\pi \;+\; %2e int%
        \frac{992\pi}{10} \int_0^{\frac{\pi}{2}} \sin\phi(1-\cos^2\phi)\; d\phi = \\
    30\pi \;+\; %2e int$
        \frac{992\pi}{10} \Big(\int_0^{\frac{\pi}{2}} \sin\phi\; d\phi - 
        \int_0^{\frac{\pi}{2}} \sin\phi\cos^2\phi\; d\phi\Big) \\
    \end{multline*}
Par changement de variable en posant $u = \cos\phi$ et en calculant 
    $-du= \sin\phi \; d\phi$ :
\begin{multline*}
    J_2 = 30\pi \;+\; %2e int$
        \frac{992\pi}{10} \Big(1\; + \; \int_1^0 u^2\; du\Big ) = 
    30\pi + \frac{992\pi}{10}\big(1-\frac{1}{3}\big) = \frac{1442\pi}{15} \\
\end{multline*}

%-----------------------------QUESTION 3 ----------------------------------%
\newpage \section*{Question 3}


\begin{enumerate}[a)]

\item % a)
Soit $P = \big(\frac{1}{2}, \frac{\sqrt{3}}{2}, -\frac{1}{2}\big)$ et $\vec{r}(t)
    = \cos(\pi t)\vec{i} + \sin(\pi t)\vec{j} + \cos(2\pi t)\vec{k}, 0\le t\le 2$.
    En isolant $t$ : 
\begin{align*}
    \cos(\pi t) = \frac{1}{2} \Rightarrow t\in\big\{\frac{1}{3},\,\frac{5}{3}
        \big\},\; 0 \le \pi t \le 2\pi \\
    \sin(\pi t) = \frac{\sqrt{3}}{2} \Rightarrow t\in\big\{\frac{1}{3},\,
        \frac{2}{3}\big\},\; 0 \le \pi t \le 2\pi \\
    \cos(2\pi t) = -\frac{1}{2} \Rightarrow t\in \big\{\frac{1}{3},\,
        \frac{2}{3}\big\},\; 0 \le 2\pi t \le 2\pi \\
\end{align*}
Puisque $t=\frac{1}{3}$ est commun aux trois équations et que $t=\frac{1}{3}
    \in[0,2]$, $P \in C$

\item % b)
\[ \vec{r}^{\;\prime}(t) = -\pi\sin(\pi t)\vec{i} + \pi\cos(\pi t)\vec{j} - 
    2\pi\sin(2\pi t)\vec{k}\]

    En évaluant la dérivée au point $t = \frac{1}{3}$ :
\[ \vec{r}^{\;\prime}\Big(\frac{1}{3}\Big) = -\pi\sin\Big(\frac{\pi}{3}\Big)\vec{i} + 
    \pi\cos\Big(\frac{\pi}{3}\Big)\vec{j} - 2\pi\sin\Big(\frac{2\pi}{3}\Big)\vec{k}
    = -\frac{\sqrt{3}\pi}{2}\vec{i} + \frac{\pi}{2}\vec{j} - \sqrt{3}\pi\vec{k}\]

Pour obtenir les équations paramétriques, on ajoute le point $P$ à chaque paramètre :

\[\begin{cases}
    x=-\frac{\sqrt{3}\pi}{2}t + \frac{1}{2} \\
    y=\frac{\pi}{2}t + \frac{\sqrt{3}}{2} \\
    z=-\sqrt{3}\pi t - \frac{1}{2}
\end{cases}
0 \le t \le 2\]
\end{enumerate}



%-----------------------------QUESTION 4 ---------------------------------%
\newpage \section*{Question 4}



\begin{enumerate}[a)]

\item % a)

\begin{enumerate}[(i)]

\item % (i)

Il suffit de démontrer que les deux courbes satisfont l'équation cartésienne
    d'une sphère, soit $S:= (x(t))^2+(y(t))^2+(z(t))^2=\rho^2$. Pour $C_1$:

\begin{gather*}
    (\sqrt{3}\cos(t))^2 + (\sqrt{3}\sin(t))^2 + 1^2 = \rho^2\\
    3\cos^2(t) + 3\sin^2(t) + 1 = \rho^2 \\
    3(\cos^2(t)+\sin^2(t)) + 1 = \rho^2 \\
    3 + 1 = 4 = \rho^2 \\
    \rho = \sqrt{4} = 2\;\rho \ge 0 \\
\end{gather*}

Pour $C_2$:
\begin{gather*}
    (\sqrt{2}\cos(t))^2 + (-\sqrt{2}\cos(t))^2 + (2\sin(t))^2 = \rho^2 \\
    2\cos^2(t) + 2\cos^2(t) + 4\sin^2(t) = \rho^2 \\
    4\cos^2(t) + 4\sin^2(t) = \rho^2 \\ 
    4(\cos^2(t) + \sin^2(t)) = \rho^2) \\
    4 = \rho^2 \\
    \rho = \sqrt{4} = 2\;\rho \ge 0 \\
\end{gather*}

Ainsi, $C_1$ et $C_2$ satisfont la sphère d'équation $S:= x^2+y^2+z^2=4$, 
    soit la sphère de rayon $2$ centrée à l'origine $(0,0,0)$. 
    Tel que démontré, $\{C_1,C_2\}\in S$. \\

\item % (ii)
D'une part, il faut trouver le(s) point(s) d'intersection entre les deux courbes, soit
    s'il $\exists\,t_1,t_2\,|\,\vec{r_1}(t_1) = \vec{r_2}(t_2)$ : \\

Pour $z\,(\vec{k})$ : 
\begin{gather*}
    1 = 2\sin(t_2) \Leftrightarrow \sin(t_2) = \frac{1}{2} \Rightarrow 
        t_2 = \big\{\frac{\pi}{6}, \frac{5\pi}{6}\big\}
\end{gather*}

Pour $x\,(\vec{i})$ considérant $t_2 = \big\{\frac{\pi}{6}, \frac{5\pi}{6}\big\}$: 
\begin{gather*}
    \sqrt{3}\cos(t_1) = \sqrt{2}\cos(t_2) \\
    t_1 = \big\{\frac{\pi}{4}, \frac{7\pi}{4}\big\},\; t_2=\frac{\pi}{6} \\
    t_1 = \big\{\frac{3\pi}{4}, \frac{5\pi}{4}\big\},\; t_2=\frac{5\pi}{6} \\
\end{gather*}

Pour $y\,(\vec{j})$ considérant $t_2 = \big\{\frac{\pi}{6}, \frac{5\pi}{6}\big\}$: 
\begin{gather*}
    \sqrt{3}\sin(t_1) = -\sqrt{2}\cos(t_2) \\
    t_1 = \big\{\frac{5\pi}{4}, \frac{7\pi}{4}\big\},\; t_2=\frac{\pi}{6} \\
    t_1 = \big\{\frac{\pi}{4}, \frac{3\pi}{4}\big\},\; t_2=\frac{5\pi}{6} \\
\end{gather*}

En choisisant les angles communs, nous avons :
\begin{gather*}
    t_1 = \frac{3\pi}{4},\,t_2=\frac{5\pi}{6} \\
    t_1 = \frac{7\pi}{4},\,t_2=\frac{\pi}{6} \\
\end{gather*}

Ensuite, il faut obtenir les vecteurs tangents avec la dérivée : 
\begin{gather*}
    \vec{T_1}(t) = \vec{r_1}\,'(t) = -\sqrt{3}\sin(t_1)\vec{i} + 
        \sqrt{3}\cos(t_1)\vec{j} \\
    \vec{T_2}(t) = \vec{r_2}\,'(t) = -\sqrt{2}\sin(t_2)\vec{i} +
        \sqrt{2}\sin(t_2)\vec{j} + 2\cos(t_2)\vec{k} 
\end{gather*} 

Pour les deux points d'intersection :
\begin{gather*}
    \vec{T_1}\big(\frac{3\pi}{4}\big) = -\frac{\sqrt{6}}{2}\vec{i} - 
        \frac{\sqrt{6}}{2}\vec{j} \\
    \vec{T_2}\big(\frac{5\pi}{6}\big) = -\frac{\sqrt{2}}{2}\vec{i} + 
    \frac{\sqrt{2}}{2}\vec{j} - \sqrt{3}\vec{k} \\
    \vec{T_1}\cdot\vec{T_2} = -\frac{\sqrt{6}}{2}\cdot-\frac{\sqrt{2}}{2}
        + -\frac{\sqrt{6}}{2}\cdot\frac{\sqrt{2}}{2} + 
        0\cdot- \sqrt{3} = 0 \\
\end{gather*}

\begin{gather*}
    \vec{T_1}\big(\frac{7\pi}{4}\big) = \frac{\sqrt{6}}{2}\vec{i} + 
        \frac{\sqrt{6}}{2}\vec{j} \\
    \vec{T_2}\big(\frac{\pi}{6}\big) = -\frac{\sqrt{2}}{2}\vec{i} + 
    \frac{\sqrt{2}}{2}\vec{j} + \sqrt{3}\vec{k} \\
    \vec{T_1}\cdot\vec{T_2} = \frac{\sqrt{6}}{2}\cdot-\frac{\sqrt{2}}{2}
        + \frac{\sqrt{6}}{2}\cdot\frac{\sqrt{2}}{2} + 
        0\cdot \sqrt{3} = 0 \\
\end{gather*}

Puisque pour les deux points d'intersection $\vec{T_1}\cdot\vec{T_2} = 0$, les deux
    courbes se coupent à angle droit. \\

\end{enumerate}

\newpage
\item % b)
Soit $\vec{r}(t)$ le vecteur position et $\vec{v}(t) = \frac{d}{dt}\,\vec{r}(t)$
     le vecteur vitesse. Puisque la distance à l'origine est constante :

\[ \sqrt{x^2+y^2+z^2} = \kappa \Leftrightarrow \norme{\,\vec{r}(t)\,} = \kappa \]

Ainsi :
\[\norme{\,\vec{r}(t)\,}^2 = \kappa^2 \Leftrightarrow \vec{r}(t)\cdot
    \vec{r}(t) = \kappa^2\]

En dérivant : 
\[ \frac{d}{dt}\big(\vec{r}(t)\cdot\vec{r}(t)\big) = \frac{d}{dt}\big(\kappa^2\big) \]
\[ \vec{v}(t)\cdot\vec{r}(t) + \vec{r}(t)\cdot\vec{v}(t) = 0\]
\[ 2(\vec{v}(t)\cdot\vec{r}(t)) = 0\]
\[ \vec{v}(t)\cdot\vec{r}(t) = 0, \; \forall\,t\]

Puisque le produit scalaire de $\vec{v}(t)$ et de $\vec{r}(t)$ est nul pour tout $t$, 
les vecteurs sont orthogonaux pour tout $t$.
\end{enumerate}



%-----------------------------QUESTION 5 ---------------------------------%
\newpage \section*{Question 5}



\begin{enumerate}[a)]

\item % a)
Puisque la densité du solide est proportionnelle au carré de la distance de
    l'origine, on peut obtenir celle-ci avec l'équation $(\sqrt{x^2+y^2+z^2})^{1/2}
    = x^2+y^2+z^2$. Ainsi, $\rho(x,y,z)=\kappa(x^2+y^2+z^2)$. En coordonnées
    cylindriques, nous avons : 

\[ x^2 + y^2 = 1 \Leftrightarrow r^2 = 1 \Leftrightarrow r = 1\;, r \ge 0\]
\[ x^2 + y^2 + z^2 = 4 \Leftrightarrow r^2 + z^2 = 4 \Leftrightarrow r=\sqrt{4-z^2},
    \; r \ge 0 \]

Pour trouver les bornes de $z$, on trouve l'intersection entre la sphère et le
    cylindre :

\[ 1 = 4-z^2 \Leftrightarrow x^2 = 3 \Leftrightarrow z = \pm\sqrt{3} \]

Considérant que la sphère et le cylindre font un tour complet, $\theta\in[0,2\pi]$.
    Ainsi, nous avons la région suivante : 

\[ B_C=\{(r,\theta, z)\;|\;1\le r\le \sqrt{4-z^2},\;0\le \theta \le 2\pi,\;
    -\sqrt{3} \le z \le \sqrt{3}\} \]

Quant à l'intégrale :

\[ m = \iiint\limits_{B_C} \kappa\,(x^2+y^2+z^2)\;dV = \kappa\,\int_0^{2\pi}
    \int_{-\sqrt{3}}^{\sqrt{3}}\int_1^{\sqrt{4-z^2}} \big(r^3 + rz^2\big)\;
    drdzd\theta\] \\

En coordonnées sphériques, pour obterir les bornes de $\rho$ :

\begin{multline*}
    x^2 + y^2 = 1 \Leftrightarrow \rho^2\sin^2\phi\cos^2\theta + 
        \rho^2\sin^2\phi\sin^2\theta = 1 \Leftrightarrow \\
        \rho^2\sin^2\phi(\cos^2\theta + \sin^2\theta) = 1 \Leftrightarrow
        \rho\sin\phi = 1 \Leftrightarrow \rho=\csc\theta,\;\rho\ge 0 \\
\end{multline*}
\[ x^2 + y^2 + z^2 = 4 \Leftrightarrow \rho^2 = 4 \Leftrightarrow \rho = 2,\;\rho\ge0\]

Quant aux bornes de $\phi$, il suffit de trouver l'angle par rapport à l'axe des 
    $z$ positif où $\rho = 2$ et $z=\sqrt{3}$ (intersection trouvée lors des
    coordonnées cylindriques) :
\[ cos\phi = \frac{\sqrt{3}}{2} \Leftrightarrow \phi = 
    \arccos\Big(\frac{\sqrt{3}}{2}\Big) = \frac{\pi}{6} \]
\[ \pi - \frac{\pi}{6} = \frac{5\pi}{6}\]

Considérant que la sphère et le cylindre font un tour complet, $\theta\in[0,2\pi]$.
    Ainsi, nous avons la région suivante : 

\[ B_S=\Big\{(\rho,\theta, \phi)\;|\;\csc\phi\le \rho\le 2,\;0 \le \theta \le 2\pi,
    \; \frac{\pi}{6}\le \phi \le \frac{5\pi}{6}\Big\} \]

Quant à l'intégrale :

\[ m = \iiint\limits_{B_C} \kappa\,(x^2+y^2+z^2)\;dV = \kappa\,\int_0^{2\pi}
    \int_{\frac{\pi}{6}}^{\frac{5\pi}{6}}\int_{\csc\phi}^{2} \rho^4\sin\phi\;
    d\rho d\phi d\theta \]

\item % b)
En utilisant les coordonnées sphériques : 
\begin{multline*}
    m = \iiint\limits_{B_C} \kappa\,(x^2+y^2+z^2)\;dV = \kappa\,\int_0^{2\pi}
        \int_{\frac{\pi}{6}}^{\frac{5\pi}{6}}\int_{\csc\phi}^{2} \rho^4\sin\phi\;
        d\rho d\phi d\theta = \\
    \kappa\Big(\int_0^{2\pi}\;d\theta\Big)\Big(
        \int_{\frac{\pi}{6}}^{\frac{5\pi}{6}}\int_{\csc\phi}^{2} \rho^4\sin\phi\;
        d\rho d\phi\Big) =
        2\kappa\pi \int_{\frac{\pi}{6}}^{\frac{5\pi}{6}} \big(\frac{32\sin\phi}{5}
        - \frac{\csc^4\phi}{5}\big)\;d\phi = \\
    2\kappa\pi \Big(\frac{32}{5}\int_{\frac{\pi}{6}}^{\frac{5\pi}{6}} \sin\phi\; 
        d\phi - \frac{1}{5}\int_{\frac{\pi}{6}}^{\frac{5\pi}{6}} \csc^4\phi\;d\phi
        \Big) \\
\end{multline*}
En utilisant la formule de réduction (\#78) à deux reprises :
\begin{multline*}
    m = 2\kappa\pi \Big(\frac{32}{5}\cdot\sqrt{3} - \frac{1}{5}\big(
        \frac{1}{3}\cot\phi\csc^2\phi\Big|_{\frac{5\pi}{6}}^{\frac{\pi}{6}} +
        \frac{2}{3} \int_{\frac{\pi}{6}}^{\frac{5\pi}{6}} \csc^2\theta d\theta
        \big)\Big) = \\
    2\kappa\pi \Big(\frac{32\sqrt{3}}{5} -\frac{1}{5}\big(\frac{8\sqrt{3}}{3}
        + \frac{2}{3}\cot\phi\Big|_{\frac{5\pi}{6}}^{\frac{\pi}{6}}\big)
        \Big) = 
        2\kappa\pi \Big(\frac{32\sqrt{3}}{5} - \frac{12\sqrt{3}}{15}\Big) = 
        \kappa\,\frac{56\sqrt{3}\pi}{5} \\
\end{multline*}

\end{enumerate}

\end{document}
