%--------------------------INITIALISATION DU DOCUMENT---------------------%
%											%														%											%
%                        >>>> Ne pas modifier cette partie <<<<			%
																																					
\documentclass[letterpaper,12pt,oneside,final]{book}



%%
%%  Version: 2014-10-28
%%
%%  Accepte les caractères accentués dans le document (UTF-8).
\usepackage[utf8]{inputenc}
%%
%% Support pour l'anglais et le français (français par défaut).
%\usepackage[cyr]{aeguill}
\usepackage{lmodern}      % Police de caractères plus complète et généralement indistinguable visuellement de la police standard de LaTeX (Computer Modern).
\usepackage[T1]{fontenc}  % Bon encodage des caractères pour qu'Acrobat Reader reconnaisse les accents et les ligatures telles que ffi.
\usepackage[english,frenchb]{babel} % le langage par défaut est le dernier de la liste, c'est-à-dire français
%%
%% Charge le module d'affichage graphique.
\usepackage{graphicx}
\usepackage{epstopdf}  % Permet d'utiliser des .eps avec pdfLaTeX.
%%
%% Recherche des images dans les répertoires.
\graphicspath{{./images/}{./dia/}{./gnuplot/}}
%%
%% Un float peut apparaître seulement après sa définition, jamais avant.
\usepackage{flafter,placeins}
%%
%% Utilisation de natbib pour les citations et la bibliographie.
\usepackage{natbib}
%%
%% Autres packages.
\usepackage{amsmath,color,soulutf8,longtable,colortbl,setspace,ifthen,xspace,url,pdflscape,tikz,pgfplots}
%%
%% Support des acronymes.
\usepackage[nolist]{acronym}
\onehalfspacing                % Interligne 1.5.
%%
%% Définition d'un style de page avec seulement le numéro de page à
%% droite. On s'assure aussi que le style de page par défaut soit
%% d'afficher le numéro de page en haut à droite.
\usepackage{fancyhdr}
\fancypagestyle{pagenumber}{\fancyhf{}\fancyhead[R]{\thepage}}
\renewcommand\headrulewidth{0pt}
\makeatletter
\let\ps@plain=\ps@pagenumber
\makeatother
%%
%% Module qui permet la création des bookmarks dans un fichier PDF.
%\usepackage[dvipdfm]{hyperref}
\usepackage{hyperref}
\usepackage{caption}  % Hyperlien vers la figure plutôt que son titre.

\usepackage{esint}
\usepackage{geometry}
\usepackage{enumerate}




\begin{document}
\setlength{\parindent}{0pt}
%--------------------------------------------------------------------------------------%

%--------------------------PAGE DE COUVERTURE------------------------------%

% A REMPLIR PAR L'ETUDIANT: 

\newcommand\monPrenom{William}		%PRENOM
\newcommand\monNom{Trépanier}			%NOM
\newcommand\monMatricule{1952594}	%MATRICULE
\newcommand\monGroupe{1}		%GROUPE

%------------------------ Ne pas modifier la ligne suivante --------------%
%\newgeometry{tmargin=2.0cm, bmargin=2.0cm, lmargin=2.25cm, rmargin=2.25cm, headsep=1.0cm}
\newgeometry{top=2cm}
\definecolor{gris1}{gray}{0.75}

\newcommand{\encadre}[1]{
\setlength\fboxsep{5mm}\setlength\fboxrule{1pt}
\begin{center}
\fcolorbox{black}{gris1}{
\begin{minipage}{0.94\textwidth}{#1}\end{minipage}}
\end{center}}

% encadre blanc
\newcommand{\boite}[1]{
\setlength\fboxsep{5mm}\setlength\fboxrule{1pt}
\begin{center}
\fcolorbox{black}{white}{
\begin{minipage}{0.5\textwidth}{#1}\end{minipage}}
\end{center}}


%\begin{document}

\thispagestyle{empty}

{
\centering

\encadre{
\begin{center}
\bf
{\Large \scshape 
Polytechnique Montr\'eal
\\
D\'epartement de Math\'ematiques et de G\'enie Industriel
}
\\
{\Huge
\

MTH1102 - Calcul II
\\
\'Et\'e 2019 - Trimestre court

\

Devoir 4

}
\end{center}
}

\vfill

\fcolorbox{black}{white}{
\begin{minipage}{0.94\linewidth}

\vspace{5mm}

{\bf \Large Nom: }\monNom \hspace{20mm} {\bf \Large Pr\'enom: }\monPrenom%\rule[-1mm]{56mm}{0.6pt}

\vspace{8mm}

{\bf \Large Matricule: }\monMatricule \hspace{20mm} {\bf \Large Section: }\monGroupe

%\vspace{8mm}
%
%{\bf \Large Signature: }\rule[-1mm]{126mm}{0.6pt}
%
%
%\vspace{5mm}

\end{minipage}}

\vfill


{
\renewcommand{\arraystretch}{1.5}
\begin{center}
\begin{tabular}{|c|c|c||c|} \hline
{\bf \Large Question}							& {\bf \Large Autres}			& 	{\bf \Large Bonus}	&  \\ 
{\bf \Large corrig\'ee}						& {\bf \Large  questions}	&		{\bf \Large \LaTeX}	& {\bf \Large Total} \\ \hline
\hspace{20mm}			{\Huge \strut}	& \hspace{20mm}						&		\hspace{20mm}				&\hspace{20mm} \\
\hspace{20mm}			{\Huge \strut}	& \hspace{20mm}						&		\hspace{20mm} 			& \hspace{20mm} {\Large /10} \\ \hline
\end{tabular}
\end{center}
}


\vfill

}

\restoregeometry
%\end{document}
%-------------------------------------------------------------------------%


%========================= Début des réponses ============================%


%-----------------------------QUESTION 1 ----------------------------------%
\section*{Question 1}

Le paraboloïde $z_1=x^2+y^2+2$ se retrouve au dessus du paraboloïde 
    $z_2=2x^2+2y^2$. En passant aux coordonnées cylindriques, on obtient :
    $z_1 = r^2 + 2$ et $z_2 = 2r^2$. Pour trouver les bornes de $r$ sur sa 
    projection dans le plan $xy$, il faut trouver l'intersection des deux
    paraboloïdes :

\[ 2r^2 = r^2 + 2 \Leftrightarrow r^2-2 = 0 \Leftrightarrow 
    (r+\sqrt{2})(r-\sqrt{2}) = 0 \Rightarrow r = \sqrt{2},\; r \ge 0 \]

Ainsi, nous obtenons :

\[ E=\{(r,\theta, z)\;|\;0\le r\le \sqrt{2},\;0\le \theta \le 2\pi,\;
    2r^2 \le z \le r^2+2\} \]

Quant à l'intégrale : 

\begin{multline*}
    J_1 = \iiint\limits_E (x^2+y^2)^{3/2}\;dV = \int_0^{2\pi} \int_0^{\sqrt{2}}
        \int_{2r^2}^{r^2+2} (r^2)^{3/2}\cdot r\;dzdrd\theta = \\
        \int_0^{2\pi} \int_0^{\sqrt{2}} \int_{2r^2}^{r^2+2} r^4\;dzdrd\theta =
        \int_0^{2\pi} \int_0^{\sqrt{2}} \big(2r^4 - r^6\big)\;drd\theta =
        \int_0^{2\pi} \Big(2^{5/2}\big(\frac{2}{5}-\frac{2}{7}\big)\Big)\;
            d\theta = \\
        2\pi\cdot\frac{16\sqrt{2}}{35} = \frac{32\sqrt{2}\pi}{35}
\end{multline*}

%-----------------------------QUESTION 2 ----------------------------------%
\newpage\section*{Question 2}


% A REMPLIR PAR L'ETUDIANT:


%-----------------------------QUESTION 3 ----------------------------------%
\newpage \section*{Question 3}


\begin{enumerate}[a)]

\item % a)

 % A REMPLIR PAR L'ETUDIANT:

\item % b)

 % A REMPLIR PAR L'ETUDIANT:


\end{enumerate}



%-----------------------------QUESTION 4 ---------------------------------%
\newpage \section*{Question 4}



\begin{enumerate}[a)]

\item % a)

\begin{enumerate}[(i)]

\item % (i)

 % A REMPLIR PAR L'ETUDIANT:

\item % (ii)

\end{enumerate}

\item % b)

 % A REMPLIR PAR L'ETUDIANT:

\end{enumerate}



%-----------------------------QUESTION 5 ---------------------------------%
\newpage \section*{Question 5}



\begin{enumerate}[a)]

\item % a)

 % A REMPLIR PAR L'ETUDIANT:

\item % b)

 % A REMPLIR PAR L'ETUDIANT:


\end{enumerate}



\end{document}
