%--------------------------INITIALISATION DU DOCUMENT---------------------%
%											%														%											%
%                        >>>> Ne pas modifier cette partie <<<<			%
																																					
\documentclass[letterpaper,12pt,oneside,final]{book}



%%
%%  Version: 2014-10-28
%%
%%  Accepte les caractères accentués dans le document (UTF-8).
\usepackage[utf8]{inputenc}
%%
%% Support pour l'anglais et le français (français par défaut).
%\usepackage[cyr]{aeguill}
\usepackage{lmodern}      % Police de caractères plus complète et généralement indistinguable visuellement de la police standard de LaTeX (Computer Modern).
\usepackage[T1]{fontenc}  % Bon encodage des caractères pour qu'Acrobat Reader reconnaisse les accents et les ligatures telles que ffi.
\usepackage[english,frenchb]{babel} % le langage par défaut est le dernier de la liste, c'est-à-dire français
%%
%% Charge le module d'affichage graphique.
\usepackage{graphicx}
\usepackage{epstopdf}  % Permet d'utiliser des .eps avec pdfLaTeX.
%%
%% Recherche des images dans les répertoires.
\graphicspath{{./images/}{./dia/}{./gnuplot/}}
%%
%% Un float peut apparaître seulement après sa définition, jamais avant.
\usepackage{flafter,placeins}
%%
%% Utilisation de natbib pour les citations et la bibliographie.
\usepackage{natbib}
%%
%% Autres packages.
\usepackage{amsmath,color,soulutf8,longtable,colortbl,setspace,ifthen,xspace,url,pdflscape,tikz,pgfplots}
%%
%% Support des acronymes.
\usepackage[nolist]{acronym}
\onehalfspacing                % Interligne 1.5.
%%
%% Définition d'un style de page avec seulement le numéro de page à
%% droite. On s'assure aussi que le style de page par défaut soit
%% d'afficher le numéro de page en haut à droite.
\usepackage{fancyhdr}
\fancypagestyle{pagenumber}{\fancyhf{}\fancyhead[R]{\thepage}}
\renewcommand\headrulewidth{0pt}
\makeatletter
\let\ps@plain=\ps@pagenumber
\makeatother
%%
%% Module qui permet la création des bookmarks dans un fichier PDF.
%\usepackage[dvipdfm]{hyperref}
\usepackage{hyperref}
\usepackage{caption}  % Hyperlien vers la figure plutôt que son titre.

\usepackage{esint}
\usepackage{geometry}
\usepackage{enumerate}


\setlength{\parindent}{0pt}
\newcommand{\norme}[1]{\left\Vert #1\right\Vert}

\begin{document}
%--------------------------------------------------------------------------------------%

%--------------------------PAGE DE COUVERTURE------------------------------%

% A REMPLIR PAR L'ETUDIANT: 

\newcommand\monPrenom{William}		%PRENOM
\newcommand\monNom{Trépanier}			%NOM
\newcommand\monMatricule{1952594}	%MATRICULE
\newcommand\monGroupe{1}		%GROUPE

%------------------------ Ne pas modifier la ligne suivante --------------%
%\newgeometry{tmargin=2.0cm, bmargin=2.0cm, lmargin=2.25cm, rmargin=2.25cm, headsep=1.0cm}
\newgeometry{top=2cm}
\definecolor{gris1}{gray}{0.75}

\newcommand{\encadre}[1]{
\setlength\fboxsep{5mm}\setlength\fboxrule{1pt}
\begin{center}
\fcolorbox{black}{gris1}{
\begin{minipage}{0.94\textwidth}{#1}\end{minipage}}
\end{center}}

% encadre blanc
\newcommand{\boite}[1]{
\setlength\fboxsep{5mm}\setlength\fboxrule{1pt}
\begin{center}
\fcolorbox{black}{white}{
\begin{minipage}{0.5\textwidth}{#1}\end{minipage}}
\end{center}}


%\begin{document}

\thispagestyle{empty}

{
\centering

\encadre{
\begin{center}
\bf
{\Large \scshape 
Polytechnique Montr\'eal
\\
D\'epartement de Math\'ematiques et de G\'enie Industriel
}
\\
{\Huge
\

MTH1102 - Calcul II
\\
\'Et\'e 2019 - Trimestre court

\

Devoir 4

}
\end{center}
}

\vfill

\fcolorbox{black}{white}{
\begin{minipage}{0.94\linewidth}

\vspace{5mm}

{\bf \Large Nom: }\monNom \hspace{20mm} {\bf \Large Pr\'enom: }\monPrenom%\rule[-1mm]{56mm}{0.6pt}

\vspace{8mm}

{\bf \Large Matricule: }\monMatricule \hspace{20mm} {\bf \Large Section: }\monGroupe

%\vspace{8mm}
%
%{\bf \Large Signature: }\rule[-1mm]{126mm}{0.6pt}
%
%
%\vspace{5mm}

\end{minipage}}

\vfill


{
\renewcommand{\arraystretch}{1.5}
\begin{center}
\begin{tabular}{|c|c|c||c|} \hline
{\bf \Large Question}							& {\bf \Large Autres}			& 	{\bf \Large Bonus}	&  \\ 
{\bf \Large corrig\'ee}						& {\bf \Large  questions}	&		{\bf \Large \LaTeX}	& {\bf \Large Total} \\ \hline
\hspace{20mm}			{\Huge \strut}	& \hspace{20mm}						&		\hspace{20mm}				&\hspace{20mm} \\
\hspace{20mm}			{\Huge \strut}	& \hspace{20mm}						&		\hspace{20mm} 			& \hspace{20mm} {\Large /10} \\ \hline
\end{tabular}
\end{center}
}


\vfill

}

\restoregeometry
%\end{document}
%-------------------------------------------------------------------------%


%========================= Début des réponses ============================%


%-----------------------------QUESTION 1 ----------------------------------%
\section*{Question 1}

En utilisant la formule et soit D la région située dans le premier octant 
    et bornée par le paraboloïde $z = 2(x^2+y^2)$ et le plan $z=2$:
\begin{gather*}
    A = \iint\limits_D \sqrt{1 + \Big(\frac{\partial z}{\partial x}\Big)^2 + 
        \Big(\frac{\partial z}{\partial y}\Big)^2}\,dA =
        \iint\limits_D \sqrt{1+(4x)^2+(4y)^2}\,dA \\
    = \iint\limits_D \sqrt{1+16(x^2+y^2)}\,dA = \int_0^{\pi/2} 
        \int_0^1 \sqrt{1 + 16r^2}\,r\,drd\theta \\
    = \int_0^{\pi/2} d\theta \int_0^1 r\,\sqrt{1 + 16r^2}\,drd\theta
        = \frac{\pi}{2} \int_0^1 r\,\sqrt{1 + 16r^2}\,drd\theta
\end{gather*}
En posant $u = 1+16r^2$ et en calculant $du=32r\,dr$:
\begin{gather*}
    A = \frac{\pi}{64} \int_1^{17} u^{1/2}\,du = 
        \frac{\pi}{96}\, u^{3/2}\Big|_1^{17} =
        \frac{\pi}{96}\, (17^{\frac{3}{2}} - 1) = 
        \frac{\pi}{96}\, (17\sqrt{17} - 1) 
\end{gather*}


%-----------------------------QUESTION 2 ----------------------------------%
\newpage\section*{Question 2}
On trouve d'abord $\vec{r_u}$ et $\vec{r_v}$:
\begin{gather*}
    \vec{r_u} = 2u\vec{i} + v\vec{j} \\
    \vec{r_v} = u\vec{j} +v\vec{k} 
\end{gather*}

On calcule ensuite $\norme{\vec{r_u} \times \vec{r_v}}$ : 
\begin{gather*}
    \vec{r_u}\times\vec{r_v} = 
        \begin{vmatrix}
            \vec{i} & \vec{j} & \vec{k} \\
            2u      & v       & 0       \\
            0       & u       & v      
        \end{vmatrix}
        = v^2\vec{i} - 2uv\vec{j} + 2u^2\vec{k} \\
    \\
    \norme{\vec{r_u} \times \vec{r_v}} = \sqrt{
        (v^2)^2 + (-2uv)^2 + (2u^2)^2} = \sqrt{v^4
        + 4u^2v^2 + 4u^4} \\
    = \sqrt{(v^2 + 2u^2)^2} = v^2 + 2u^2 
\end{gather*}

Quant à l'intégrale et soit D le domaine des bornes de $u,v$ :
\begin{gather*}
    \iint\limits_S zx\,dS = \frac{1}{2}\iint\limits_D (v^2u^2)(v^2+2u^2)\,dA
        = \frac{1}{2}\int_{-2}^2\int_0^1 (u^2v^4+2u^4v^2)\,dvdu \\
    = \frac{1}{2}\int_{-2}^2 \Big[\frac{u^2v^5}{5}+\frac{2u^4v^3}{3}\Big]
        _{v=0}^{v=1}\,du = \frac{1}{2}\int_{-2}^2 \Big(\frac{u^2}{5} + 
        \frac{2u^4}{3} \Big)\,du \\
    = \frac{1}{2} \Big[\frac{u^3}{15} + \frac{2u^5}{15}\Big]_{u=-2}^{u=2}
        = \frac{1}{2} \cdot \frac{144}{15} = \frac{25}{4}
\end{gather*}

%-----------------------------QUESTION 3 ----------------------------------%
\newpage \section*{Question 3}
On trouve d'abord $\vec{r_u}$ et $\vec{r_v}$:
\begin{gather*}
    \vec{r_u} \times \vec{r_v} = 
        \begin{vmatrix}
            \vec{i} & \vec{j} & \vec{k} \\
            0       & 2u      & 1       \\
            1       & -1      & 2v 
        \end{vmatrix}
        = (4uv+1)\vec{i} + \vec{j} -2u\vec{k}
\end{gather*}

On vérifie ensuite l'orientation avec le vecteur normal au point $(2,2,2)$:
\begin{gather*}
    x = v \Rightarrow v = 2 \\
    z = u+2v \Rightarrow 2 = u + 4 \Rightarrow u = -2 \\
    y = 4 - 2 = 2 \\
    \\
    \vec{n}(-2,2) = \frac{(\vec{r_u}\times\vec{r_v})(-2,2)}{\norme{
        (\vec{r_u}\times\vec{r_v})(-2,2)}} = \frac{-15\vec{i}+\vec{j}+4\vec{k}}
        {\sqrt{15^2+1^2+4^2}} = \frac{1}{11\sqrt{2}}(-15\vec{i}+\vec{j}+4\vec{k})
\end{gather*}

Ainsi : 
\begin{gather*}
    \vec{r_u}\times\vec{r_v} = -(4uv+1)\vec{i} - \vec{j} +2u\vec{k}
\end{gather*}

Quant à l'intégrale et soit D le domaine des bornes de $u,v$ :
\begin{gather*}
    \iint\limits_S \vec{F}\cdot d\vec{S} = \iint\limits_D \vec{F}\cdot
        (\vec{r_u}\times\vec{r_v})\, dA \\
    = \iint\limits_D (v^2\vec{i}+(u+v^2)\vec{j}+(1-(u^2-v))\vec{k})\cdot
        (-(4uv+1)\vec{i} - \vec{j} +2u\vec{k}) \, dA \\
    = \int_{-3}^{3}\int_{-3}^{3} (-4uv^3-2v^2-3u+2u^3+2uv)\,dudv \\
    = \int_{-3}^{3} \Big[-2u^2v^3-2uv^2-\frac{3u^2}{2}+\frac{u^4}{2}+u^2v
        \Big]_{u=-3}^{u=3}\,dv = \int_{-3}^3 -12v^2 \,dv \\
    = -4\cdot v^3\Big|_{-3}^3 = -216 
\end{gather*}

%-----------------------------QUESTION 4 ---------------------------------%
\newpage \section*{Question 4}
\begin{enumerate}[a)]

\item % a)
Puisque pour tout point appartenant à $S$, $\vec{F}$ est tangent à $S$ et 
    considérant que $\vec{n}$ est orthogonal à $S$:
\begin{gather*}
    \angle(\vec{F}, \vec{n}) = \frac{\pi}{2} 
\end{gather*}

Ainsi et en chaque point : 
\begin{gather*}
    \iint\limits_S\vec{F}\cdot\,d\vec{S}=\iint\limits_S\vec{F}\cdot\vec{n}\,dS
        = \iint\limits_S \norme{\vec{F}} \norme{\vec{n}} 
        \cos(\angle(\vec{F}, \vec{n}))\,dS = \iint\limits_S 0\,dS = 0
\end{gather*}

\item % b)
Puisque $\norme{\vec{n}} = 1$ :
\begin{gather*}
    \iint\limits_S\vec{F}\cdot\,d\vec{S}=\iint\limits_S\vec{F}\cdot\vec{n}\,dS
        =  c\iint\limits_S \vec{n}\cdot\vec{n}\,dS
        =  c\iint\limits_S \norme{\vec{n}}^2\,dS \\
    =  c\iint\limits_S 1^2\,dS = cA
\end{gather*}


\end{enumerate}



%-----------------------------QUESTION 5 ---------------------------------%
\newpage \section*{Question 5}



\begin{enumerate}[a)]

\item % a)
D'une part, il suffit de démontrer que $\vec{r}(0) = \vec{r}(2\pi)$:
\begin{gather*}
    \vec{r}(0) = 10\cos(0)\vec{i} + 10\sin(0)\vec{j} + 100\cos^2(0)\vec{k}
        = 10\vec{i} + 100\vec{k} \\
    \vec{r}(2\pi) = 10\cos(2\pi)\vec{i} + 10\sin(2\pi)\vec{j} + 
        100\cos^2(2\pi)\vec{k} = 10\vec{i} + 100\vec{k} \\
    \vec{r}(0) = \vec{r}(2\pi) \Rightarrow \text{C est une courbe fermée}
\end{gather*}

Nous avons $x=10\cos(t)$ et $z=100\cos^2(t) = (10\cos(t))^2 = x^2$. Ainsi :
\begin{gather*}
    z = f(x,y) = x^2
\end{gather*}

\item % b)
\begin{itemize}
    \item[$\bullet$] Il est possible de calculer directement la circulation 
        en évaluant la composante de $\vec{F}$ dans la direction du vecteur 
        tangent unitaire $\vec{T}$. Ainsi, on calcule pour obtenir la 
        circulation :  $\oint\limits_C \vec{F}\cdot\vec{T}\,ds$. \\
    \item[$\bullet$] Il est également possible d'utiliser le théorème de
        Stokes pour évaluer cette même intégrale en évaluant la composante
        normale du rotationnel de $\vec{F}$. Ainsi, on calcule pour obtenir
        la circulation: $\iint\limits_S \text{rot}\vec{F}\cdot\vec{n}\,dS$
\end{itemize}

\item % c)
Soit $C$ la courbe paramétrée, $S$ la surface paramétrée par $\vec{R}(x,y)$
    trouvée en a) et $D$ la projection de S sur le plan $xy$:
\begin{gather*}
    \oint\limits_C \vec{F}\cdot\vec{T}\,ds=\iint\limits_S\text{rot}\vec{F}
        \cdot\vec{n}\,dS = \iint\limits_D \text{rot}\vec{F}(\vec{R}(x,y))
        \cdot (\vec{R_x}\times\vec{R_y})\,dA
\end{gather*}

Nous avons : 
\begin{gather*}
    \vec{R}(x,y) = x\vec{i}+y\vec{j}+x^2\vec{k} \\
    \vec{R_x}\times\vec{R_y} = 
        \begin{vmatrix}
            \vec{i} & \vec{j} & \vec{k} \\
            1       & 0       & 2x      \\
            0       & 1       & 0
        \end{vmatrix}
        = -2x\vec{i} + \vec{k}
\end{gather*}
\begin{gather*}
    \text{rot}\vec{F} = 
        \begin{vmatrix}
            \vec{i} & \vec{j} & \vec{k}       \\
            \frac{\partial}{\partial x} & \frac{\partial}{\partial y}
                & \frac{\partial}{\partial z} \\
            y^2+z^2 & 1       & xy+\sin(z^4)
        \end{vmatrix}
        = x\vec{i} - (y-2z)\vec{j} -2y\vec{k} \\
    \text{rot}\vec{F}\cdot(\vec{R_x}\times\vec{R_y}) = -2x^2 -2y
\end{gather*}

Considérant que D est le disque de rayon $10$ centré à l'origine:
\begin{gather*}
    \iint\limits_D \text{rot}\vec{F}(\vec{R}(x,y))
        \cdot (\vec{R_x}\times\vec{R_y})\,dA = 
        \iint\limits_D (-2x^2 -2y)\,dA \\
    = \int_0^{2\pi}\int_0^{10} (-2r^2\cos^2\theta-2r\sin\theta)r\,
        drd\theta = \int_0^{2\pi}\int_0^{10} (-2r^3\cos^2\theta-
        2r^2\sin\theta)\,drd\theta \\
    = \int_0^{2\pi}\Big[-\frac{\cos^2\theta}{2}r^4 - \frac{2\sin\theta}{3}r^3
        \Big]_{r=0}^{r=10}\,d\theta = \int_0^{2\pi}\Big(-5000\cos^2\theta - 
        \frac{2000}{3}\sin\theta\Big)\,d\theta \\
    = \int_0^{2\pi} \Big(-5000\Big(\frac{1+\cos2\theta}{2}\Big)-
        \frac{2000}{3}\sin\theta\Big)\,d\theta \\
    = \int_0^{2\pi} \Big(-2500 - \frac{2500}{2}\cos\theta\ -
        \frac{2000}{3}\sin\theta\Big)\,d\theta \\
    = \Big[-2500\theta-\frac{2500}{2}\sin\theta + 
        \frac{2000}{3}\cos\theta\Big]_{0}^{2\pi} = -5000\pi
\end{gather*}

\end{enumerate}


%-----------------------------QUESTION 6 ---------------------------------%
\newpage \section*{Question 6}



\begin{enumerate}[a)]

\item % a)
Une surface fermée est une surface qui est la frontière d'une région solide
    $E$. Ainsi pour démontrer que la surface $S$ est fermée, il suffit de
    démontrer que $E$ existe : 
\begin{gather*}
    E:=\{(x,y,z)|\,0\le x\le 2,\,0\le y\le \sqrt{2},\,0\le z\le 2-y^2\} \\
    E \,\exists \Rightarrow S \text{ est fermée}
\end{gather*}

\item % b)
Puisque $\vec{n}$ pointe vers l'intérieur de la surface, l'orientation est
    négative. En utilisant le théorème de flux-divergence : 

\begin{gather*}
    \iint\limits_S \vec{F}\cdot d\vec{S} = \iiint\limits_E \text{div}\vec{F}
        \,dV \\
    \text{div}\vec{F} = -2x -2y + 1
\end{gather*}
Puisque l'orientation de la surface est négative et que le théorème de 
    flux-divergence est vrai pour les surfaces orientées positivement, 
    il suffit de multiplier $\text{div}\vec{F}\cdot -1$. Ainsi, pour 
    les fins de l'application du théorème :
\begin{gather*}
    \text{div}\vec{F} = 2x + 2y - 1
\end{gather*}

Quant à l'intégrale :
\begin{gather*}
    \iiint\limits_E \text{div}\vec{F}\,dV = \iiint\limits_E (2x+2y-1)\,dV
        = \int_0^2\int_0^{\sqrt{2}}\int_0^{2-y^2} (2x+2y-1)\,dzdydx \\
    \int_0^2\int_0^{\sqrt{2}}(2x+2y-1)(2-y^2)\,dydx = 
        \int_0^2\int_0^{\sqrt{2}} (4x-2xy^2+4y-2y^3-2+y^2)\,dydx \\
    = \int_0^2 \Big[4xy-\frac{2xy^3}{3}+2y^2-\frac{y^4}{2}-2y+\frac{y^3}{3}
        \Big]_{y=0}^{y=\sqrt{2}}\,dx = \int_0^2\Big(\frac{8\sqrt{2}x}{3}
        +2-\frac{4\sqrt{2}}{3}\Big)\,dx \\
    = \Big[\frac{8\sqrt{2}x^2}{6} + 2x - \frac{4\sqrt{2}x}{3}\Big]_{x=0}
        ^{x=2} = 4 + \frac{8\sqrt{2}}{3}
\end{gather*}
\end{enumerate}


%-----------------------------QUESTION 7 ---------------------------------%
\newpage \section*{Question 7}

\begin{enumerate}[a)]

\item % a)
Sachant que $\rho(x,y,z) = c$
\begin{gather*}
    M_{yz} = \iiint\limits_E cx\,dV = c\iiint\limits_E x\,dV
\end{gather*}

En posant :
\begin{gather*}
    \vec{F}(x,y,z) = \frac{x^2}{2}\vec{i} \\
    \text{div}\vec{F} = x
\end{gather*}

En appliquant le théorème de flux-divergence : 
\begin{gather*}
    \iiint\limits_E \text{div}\vec{F}\,dV = \iint\limits_S \vec{F}\cdot
        d\vec{S} \Longrightarrow \\
    c\iiint\limits_E x\,dV = c \iint\limits_S \Big(\frac{x^2}{2}\vec{i}\Big)
        \cdot d\vec{S}=\frac{c}{2}\iint\limits_S (x^2\vec{i})\cdot d\vec{S}
\end{gather*}

En substituant $x$ par $y$ et $z$, on peut démontrer de manière analogue que:
\begin{gather*}
    M_{xz} = \iiint\limits_E cy\,dV = c\iiint\limits_E y\,dV = 
        \frac{c}{2}\iint\limits_S (y^2\vec{j})\cdot d\vec{S} \\
    M_{xy} = \iiint\limits_E cz\,dV = c\iiint\limits_E z\,dV = 
        \frac{c}{2}\iint\limits_S (z^2\vec{k})\cdot d\vec{S}
\end{gather*}

\item % b)
Puisque le solide est symétrique par rapport à x et y et que l'intégrale 
    des fonctions sinus et cosinus évaluée entre $0$ et $2\pi$ est égale 
    à $0$ :
\begin{gather*}
    \overline{x} = 0 \\
    \overline{y} = 0 \\ 
\end{gather*}

\begin{gather*}
    \overline{z}=\frac{M_{xy}}{m}=\frac{1}{m}\iint\limits_S (z^2\vec{k})
        \cdot d\vec{S} \\
    \vec{R_u}\times\vec{R_v} = -u\sin u\cos v\vec{i} -u\sin u\sin v\vec{j} + \\
    (u^2 \sin u\cos^3 v+u^2 \sin u\cos u\sin^2v+u\sin^2 u \sin^2 v+
        u \sin^2 u \cos^2 v)\vec{k} \\
    \overline{z}= \frac{1}{m}\iint\limits_D (u^2\vec{k})\cdot (\vec{R_u}\times
        \vec{R_v})\,dA \\
    = \frac{1}{m} \iint\limits_D u^2(u^2 \sin u\cos^3 v+u^2 \sin u\cos u\sin^2v+
        u\sin^2 u \sin^2 v+ \\
    u \sin^2 u \cos^2 v)\,dA \\
    = \frac{1}{m} \int_0^{2\pi}\int_0^{\pi} (u^4 \sin u\cos^3 v+u^4 \sin u\cos u\sin^2v+
        u^3\sin^2 u \sin^2 v+ \\
    u^3 \sin^2 u \cos^2 v)\,dudv = 0
\end{gather*}

Le centre de masse est $(0,0,0)$.

\end{enumerate}





\end{document}
