%--------------------------INITIALISATION DU DOCUMENT---------------------%
%											%														%											%
%                        >>>> Ne pas modifier cette partie <<<<			%
																																					
\documentclass[letterpaper,12pt,oneside,final]{book}



%%
%%  Version: 2014-10-28
%%
%%  Accepte les caractères accentués dans le document (UTF-8).
\usepackage[utf8]{inputenc}
%%
%% Support pour l'anglais et le français (français par défaut).
%\usepackage[cyr]{aeguill}
\usepackage{lmodern}      % Police de caractères plus complète et généralement indistinguable visuellement de la police standard de LaTeX (Computer Modern).
\usepackage[T1]{fontenc}  % Bon encodage des caractères pour qu'Acrobat Reader reconnaisse les accents et les ligatures telles que ffi.
\usepackage[english,frenchb]{babel} % le langage par défaut est le dernier de la liste, c'est-à-dire français
%%
%% Charge le module d'affichage graphique.
\usepackage{graphicx}
\usepackage{epstopdf}  % Permet d'utiliser des .eps avec pdfLaTeX.
%%
%% Recherche des images dans les répertoires.
\graphicspath{{./images/}{./dia/}{./gnuplot/}}
%%
%% Un float peut apparaître seulement après sa définition, jamais avant.
\usepackage{flafter,placeins}
%%
%% Utilisation de natbib pour les citations et la bibliographie.
\usepackage{natbib}
%%
%% Autres packages.
\usepackage{amsmath,color,soulutf8,longtable,colortbl,setspace,ifthen,xspace,url,pdflscape,tikz,pgfplots}
%%
%% Support des acronymes.
\usepackage[nolist]{acronym}
\onehalfspacing                % Interligne 1.5.
%%
%% Définition d'un style de page avec seulement le numéro de page à
%% droite. On s'assure aussi que le style de page par défaut soit
%% d'afficher le numéro de page en haut à droite.
\usepackage{fancyhdr}
\fancypagestyle{pagenumber}{\fancyhf{}\fancyhead[R]{\thepage}}
\renewcommand\headrulewidth{0pt}
\makeatletter
\let\ps@plain=\ps@pagenumber
\makeatother
%%
%% Module qui permet la création des bookmarks dans un fichier PDF.
%\usepackage[dvipdfm]{hyperref}
\usepackage{hyperref}
\usepackage{caption}  % Hyperlien vers la figure plutôt que son titre.

\usepackage{esint}
\usepackage{geometry}
\usepackage{enumerate}


\usepackage{amsfonts}


\begin{document}
\setlength{\parindent}{0pt}
%--------------------------------------------------------------------------------------%

%--------------------------PAGE DE COUVERTURE------------------------------%

% A REMPLIR PAR L'ETUDIANT: 

\newcommand\monPrenom{William}		%PRENOM
\newcommand\monNom{Trépanier}			%NOM
\newcommand\monMatricule{1952594}	%MATRICULE
\newcommand\monGroupe{1}		%GROUPE

%------------------------ Ne pas modifier la ligne suivante --------------%
%\newgeometry{tmargin=2.0cm, bmargin=2.0cm, lmargin=2.25cm, rmargin=2.25cm, headsep=1.0cm}
\newgeometry{top=2cm}
\definecolor{gris1}{gray}{0.75}

\newcommand{\encadre}[1]{
\setlength\fboxsep{5mm}\setlength\fboxrule{1pt}
\begin{center}
\fcolorbox{black}{gris1}{
\begin{minipage}{0.94\textwidth}{#1}\end{minipage}}
\end{center}}

% encadre blanc
\newcommand{\boite}[1]{
\setlength\fboxsep{5mm}\setlength\fboxrule{1pt}
\begin{center}
\fcolorbox{black}{white}{
\begin{minipage}{0.5\textwidth}{#1}\end{minipage}}
\end{center}}


%\begin{document}

\thispagestyle{empty}

{
\centering

\encadre{
\begin{center}
\bf
{\Large \scshape 
Polytechnique Montr\'eal
\\
D\'epartement de Math\'ematiques et de G\'enie Industriel
}
\\
{\Huge
\

MTH1102 - Calcul II
\\
\'Et\'e 2019 - Trimestre court

\

Devoir 4

}
\end{center}
}

\vfill

\fcolorbox{black}{white}{
\begin{minipage}{0.94\linewidth}

\vspace{5mm}

{\bf \Large Nom: }\monNom \hspace{20mm} {\bf \Large Pr\'enom: }\monPrenom%\rule[-1mm]{56mm}{0.6pt}

\vspace{8mm}

{\bf \Large Matricule: }\monMatricule \hspace{20mm} {\bf \Large Section: }\monGroupe

%\vspace{8mm}
%
%{\bf \Large Signature: }\rule[-1mm]{126mm}{0.6pt}
%
%
%\vspace{5mm}

\end{minipage}}

\vfill


{
\renewcommand{\arraystretch}{1.5}
\begin{center}
\begin{tabular}{|c|c|c||c|} \hline
{\bf \Large Question}							& {\bf \Large Autres}			& 	{\bf \Large Bonus}	&  \\ 
{\bf \Large corrig\'ee}						& {\bf \Large  questions}	&		{\bf \Large \LaTeX}	& {\bf \Large Total} \\ \hline
\hspace{20mm}			{\Huge \strut}	& \hspace{20mm}						&		\hspace{20mm}				&\hspace{20mm} \\
\hspace{20mm}			{\Huge \strut}	& \hspace{20mm}						&		\hspace{20mm} 			& \hspace{20mm} {\Large /10} \\ \hline
\end{tabular}
\end{center}
}


\vfill

}

\restoregeometry
%\end{document}
%-------------------------------------------------------------------------%


%========================= Début des réponses ============================%


%-----------------------------QUESTION 1----------------------------------%
\section*{Question 1}
Soit $R:=\{(x,y)\;|\;-1\le x \le 2, \; 0\le y \le 4 \}$: \\
\[ J_1 = \iint\limits_R (y^3 -2xy)\;dA = 
\int_0^4 \int_{-1}^2 (y^3 -2xy)\;dxdy =
\int_0^4 (y^3 \cdot x\Big|_{x=-1}^{x=2} - 2y\cdot\frac{x^2}{2}\Big|_{x=-1}^{x=2}) \;dy
= \]
\[ \int_0^4 (y^3(2-(-1)) \;-\; y(2^2-(-1)^2))\;dy = 
\int_0^4 (3y^3 - 3y)\;dy = \frac{3y^4}{4} - \frac{3y^2}{2}\Big|_0^4 = \] 
\[ \frac{3(4)^4}{4} - \frac{3(4)^2}{2} - (0 - 0) = 192 - 24 = 168 \]

%-----------------------------QUESTION 2---------------------------------%
\newpage \section*{Question 2}
Pour trouver le domaine d'intégration, il faut trouver les points d'intersection
entre la parabole $y = x^2-6x+5$ et la droite $y=5$. La parabole $y = x^2-6x+5$ se
factorise en $y = (x-1)(x-5)$ ainsi ses zéros sont $x=1$ et $x=5$. Le sommet de la 
parabole est obtenue avec l'équation $\frac{-b}{2a} = \frac{6}{2\cdot1} = 3$.
Puisque $3^2 - 6\cdot3 + 5 = -4$, le sommet de la parabole est le point (3, -4). 
Quant à l'intersection entre la parabole et la droite, on pose :
\begin{equation}
	5 = x^2 - 6x + 5 \Longleftrightarrow 0 = x^2 - 6x \Longleftrightarrow 0 = x(x-6)\nonumber
\end{equation}
En évaluant $x$, on obtient les points d'intersection (0, 5) et (6, 5). La zone
d'intégration est ainsi la suivante : 
\begin{figure}[!h]
	\centering
	\includegraphics[scale=0.3]{img2_1.png}
	\captionof{figure}{Zone d'intégration}
\end{figure}

On pose ainsi $D:=\{(x,y)\;|\;0\le x \le 6, \; x^2-6x+5 \le y \le 5 \}$ \\
\[ J_2 = \iint\limits_D (x^2 + y\cos(\pi x))\; dA = \int_0^6 \int_{x^2-6x+5}^5 (x^2 + y\cos(\pi x))\; dydx = \]
\[\int_0^6 (x^2\cdot y\Big|_{x^2-6x+5}^5 + \cos(\pi x)\cdot \frac{y^2}{2}\Big|_{x^2-6x+5}^5) \;dx = \]
\[ \int_0^6 (x^2(5-x^2+6x-5) + \frac{\cos(\pi x)}{2}(5^2-(x^2-6x+5)^2) \;dx = \]
\[ \int_0^6 (-x^4 + 6x^3 - \frac{x^4\cos(\pi x)}{2} + 6x^3\cos(\pi x)
-23x^2\cos(\pi x)+30x\cos(\pi x)) \; dx = \]
\begin{multline}
	-\int_0^6 x^4\;dx \;+ \int_0^6 6x^3\;dx \;- \int_0^6 \frac{x^4\cos(\pi x)}{2} \;dx
	+ \int_0^6 6x^3\cos(\pi x)\;dx \; \\- \int_0^6 23x^2\cos(\pi x)\;dx \;
	+ \int_0^6 30x\cos(\pi x) \;dx = -I_1 + I_2 - I_3 + I_4 - I_5 + I_6 \nonumber
\end{multline}
Pour faciliter la lisibilité, les six intégrales seront évaluées séparément :
\[ I_1 = \int_0^6 x^4\;dx = \frac{x^5}{5}\Big|_0^6 =\frac{6^5}{5}-0=\frac{7776}{5}\]
\[ I_2 = \int_0^6 6x^3\;dx = 6 \int_0^6 x^3 \; dx = \frac{6 x^4}{4}\Big|_0^6 = \frac{3\cdot 6^4}{2}-0=1944\]
\[ I_3 = \int_0^6 \frac{x^4\cos(\pi x)}{2} \;dx = \frac{1}{2} \int_0^6 x^4 (\cos(\pi  x)) \; dx\]
En posant $u = x^4$ et $dv = \cos(\pi x)\;dx$, on calcule aussi $du = 4x^3\;dx$ et $v = \frac{\sin(\pi x)}{\pi}$. Ainsi, par l'intégration par parties : \\
\[ I_3 = \frac{x^4 \sin(\pi x)}{2 \pi}\Big|_0^6 - \frac{2}{\pi} \int_0^6 x^3 (\sin(\pi x)) \; dx = 
(0 - 0) - \frac{2}{\pi} \int_0^6 x^3 (\sin(\pi x)) \; dx = - \frac{2}{\pi} \int_0^6 x^3 (\sin(\pi x))\]
En posant $u = x^3$ et $dv = \sin(\pi x)\;dx$, on calcule aussi $du = 3x^2\;dx$ et $v = \frac{-\cos(\pi x)}{\pi}$. Ainsi, par l'intégration par parties : \\
\[ I_3 = \frac{2 x^3 \cos(\pi  x) )}{\pi ^2}\Big|_0^6 - \frac{6}{\pi^2} \int_0^6 x^2 \cos(\pi  x) \; dx 
= \frac{432}{\pi ^2} - \frac{6}{\pi^2} \int_0^6 x^2 \cos(\pi  x) \; dx\]
En posant $u = x^2$ et $dv = \cos(\pi x)\;dx$, on calcule aussi $du = 2x\;dx$ et $v = \frac{\sin(\pi x)}{\pi}$. Ainsi, par l'intégration par parties : \\
\[ I_3 = \frac{432}{\pi^2} - \frac{6 x^2 \sin(\pi  x)}{\pi ^3}\Big|_0^6 + \frac{12}{\pi^3} \int_0^6 x\sin(\pi  x)\;dx
= \frac{432}{\pi^2} + \frac{12}{\pi^3} \int_0^6 x\sin(\pi  x)\;dx\] \\
En posant $u = x$ et $dv = \sin(\pi x)\;dx$, on calcule aussi $du = dx$ et et $v = \frac{-\cos(\pi x)}{\pi}$. Ainsi, par l'intégration par parties : \\
\begin{multline}
	I_3 = \frac{432}{\pi^2} - \frac{12x\cos(\pi x)}{\pi^4}\Big|_0^6 + \frac{12}{\pi^4} \int_0^6 \cos(\pi x)\;dx 
	= \frac{432}{\pi^2} - \frac{72}{\pi^4} + \frac{12}{\pi^4}\int_0^6 \cos(\pi x) \;dx \\ 
	= \frac{432}{\pi^2} - \frac{72}{\pi^4} + \frac{12}{\pi^5}\int_0^{6\pi} \cos(u) \;du 
	= \frac{432}{\pi^2} - \frac{72}{\pi^4} + \frac{12 \sin(u)}{\pi^5}\Big|_0^{6\pi} 
	= \frac{432}{\pi^2} - \frac{72}{\pi^4}
\nonumber
\end{multline}
\[ I_4 = \int_0^6 6x^3 \cos(\pi x) \; dx = 6\int_0^6 x^3 \cos(\pi x) \; dx\]
En posant $u = x^3$ et $dv = \cos(\pi x)\;dx$, on calcule aussi $du = 3x^2\;dx$ et $v = \frac{\sin(\pi x)}{\pi}$. Ainsi, par l'intégration par parties : \\
\[ I_4 = \frac{6x^3\sin(\pi x)}{\pi}\Big|_0^6 - \frac{18}{\pi}\int_0^6 x^2 \sin(\pi x) \;dx = - \frac{18}{\pi}\int_0^6 x^2 \sin(\pi x) \;dx\]
En posant $u = x^2$ et $dv = \sin(\pi x)\;dx$, on calcule aussi $du = 2x\;dx$ et $v = \frac{-\cos(\pi x)}{\pi}$. Ainsi, par l'intégration par parties : \\
\[ I_4 = \frac{18x^2\cos(\pi x)}{\pi^2}\Big|_0^6 - \frac{36}{\pi^2}\int_0^6 x\cos(\pi x) \;dx = 
\frac{648}{\pi^2} - \frac{36}{\pi^2}\int_0^6 x\cos(\pi x) \;dx \]
En posant $u = x$ et $dv = \cos(\pi x)\;dx$, on calcule aussi $du = dx$ et $v = \frac{\sin(\pi x)}{\pi}$. Ainsi, par l'intégration par parties : \\
\begin{multline}
	I_4 = \frac{648}{\pi^2} - \frac{36x\sin(\pi x)}{\pi^3}\Big|_0^6 + \frac{36}{\pi^3}\int_0^6 \sin(\pi x) \; dx =
	\frac{648}{\pi^2} + \frac{36}{\pi^4}\int_0^{6\pi} \sin(u)\; du \\
	= \frac{648}{\pi^2} - \frac{36 \cos(u)}{\pi^4}\Big|_0^{6\pi} = \frac{648}{\pi^2}
	\nonumber
\end{multline}
\[I_5 = \int_0^6 23x^2\cos(\pi x)\;dx = 23 \int_0^6 x^2\cos(\pi x)\;dx \]
En posant $u = x^2$ et $dv = \cos(\pi x)\;dx$, on calcule aussi $du = 2x\;dx$ et $v = \frac{\sin(\pi x)}{\pi}$. Ainsi, par l'intégration par parties : \\
\[I_5 = \frac{23x^2\sin(\pi x)}{\pi}\Big|_0^6 - \frac{46}{\pi}\int_0^6 x\sin(\pi x) \; dx
= - \frac{46}{\pi}\int_0^6 x\sin(\pi x) \; dx \]
En posant $u = x$ et $dv = \sin(\pi x)\;dx$, on calcule aussi $du = dx$ et et $v = \frac{-\cos(\pi x)}{\pi}$. Ainsi, par l'intégration par parties : \\
\begin{multline}
	I_5 = \frac{46x\cos(\pi x)}{\pi^2}\Big|_0^6 - \frac{46}{\pi^2}\int_0^6 \cos(\pi x) \;dx = \frac{276}{\pi^2} - \frac{46}{\pi^2}\int_0^6 \cos(\pi x) \;dx \\
	= \frac{276}{\pi^2} - \frac{46}{\pi^3}\int_0^{6\pi} \cos(u)\;du = \frac{276}{\pi^2} - \frac{46 \sin(u)}{\pi^3}\Big|_0^{6\pi} = \frac{276}{\pi^2}
	\nonumber 
\end{multline}
\[ I_6 = \int_0^6 30x\cos(\pi x) \;dx = 30 \int_0^6 x\cos(\pi x) \;dx\]
En posant $u = x$ et $dv = \cos(\pi x)\;dx$, on calcule aussi $du = dx$ et $v = \frac{\sin(\pi x)}{\pi}$. Ainsi, par l'intégration par parties : \\
\[I_6 = \frac{30x\sin(\pi x)}{\pi}\Big|_0^6 - \frac{30}{\pi}\int_0^6 \sin(\pi x) \;dx = \frac{-30}{\pi^2}\int_0^{6\pi} \sin(u)\; du = \frac{30 \cos(u)}{\pi^2}\Big|_0^{6\pi} = 0 \]
Ainsi :
\[ J_2 = -\frac{7776}{5} + 1944 - \frac{432}{\pi^2} + \frac{72}{\pi^4} + \frac{648}{\pi^2} - \frac{276}{\pi^2} + 0 = \frac{1944}{5} - \frac{60}{\pi^2} + \frac{72}{\pi^4}\]

%-----------------------------QUESTION 3---------------------------------%
\newpage \section*{Question 3}
Pour calculer l'intégrale, il est plus simple de passer aux coordonnées polaires. 
La droite $y=x$ coupe le premier cadran en deux. Ainsi, $\theta\in [0, \frac{\pi}{4}]$. Également,
$r = \sqrt{9} = 3$. On pose $D:=\{(r,\theta)\;|\;0\le r\le 3,\;0\le \theta \le \frac{\pi}{4}\}$ : 
\begin{figure}[!h]
	\centering
	\includegraphics[scale=0.3]{img3_1.png}
	\captionof{figure}{Zone d'intégration}
\end{figure}
\begin{multline}
	J_3 = \iint\limits_D x(x^2+y^2)^{3/2}\;dA = \int_0^{\frac{\pi}{4}} \int_0^3 r\cos(\theta)(r^2)^{3/2} \;r \;drd\theta = 
	\int_0^{\frac{\pi}{4}} \int_0^3 r^5 \cos(\theta)\;drd\theta = \\
	\int_0^{\frac{\pi}{4}} \frac{r^6}{6}\Big|_0^3 \cos(\theta)\;d\theta = 
	\frac{243}{2} \int_0^{\frac{\pi}{4}} \cos(\theta) \; d\theta = \frac{243}{2} \cdot \sin(\theta)\Big|_0^{\frac{\pi}{4}} =
	\frac{243}{2}(\frac{\sqrt{2}}{2} - 0) = \frac{243\sqrt{2}}{4}
	\nonumber
\end{multline}

%-----------------------------QUESTION 4---------------------------------%
\newpage \section*{Question 4}
Il est opportun de passer des coordonnées cartésiennes aux coordonnées polaires. Le domaine d'intégration est un 
disque de rayon $\sqrt{4} = 2$, ainsi $r\in[0,2]$. Puisqu'on intègre sur le cercle en entier, $\theta\in[0, 2\pi]$. \newline
On pose $D:=\{(r,\theta)\;|\;0\le r\le 2,\;0\le \theta \le 2\pi\}$ :
\[ \iint\limits_D 10 000 - e^{(12+x^2+y^2)^{3/4}}\;dA = \int_0^{2\pi} \int_0^2 (10 000 - e^{(12+r^2)^{3/4}})r \;drd\theta\]
Soit $f(r,\theta) = (10 000 - e^{(12+r^2)^{3/4}})r$, il est possible de borner $f(r,\theta)$ par $g(r,\theta) = 7 000r$ et 
$h(r,\theta) = 10 000r$. Considérant :
\[ g(r,\theta) \le f(r, \theta) \le h(r,\theta) \quad \forall (r, \theta) \in D \]
Ainsi, par théorème :
\[ \iint\limits_D g(x,y) \;dA \le \iint\limits_D f(x,y) \;dA \le \iint\limits_D h(x,y) \;dA\]
\[ \int_0^{2\pi} \int_0^2 7 000r \;drd\theta\le \iint\limits_D f(x,y) \;dA \le \int_0^{2\pi} \int_0^2 10 000r \;drd\theta\]
\[ \int_0^{2\pi} \frac{7000r^2}{2}\Big|_0^2 \;d\theta \le \iint\limits_D f(x,y) \;dA \le \int_0^{2\pi} \frac{10000r^2}{2}\Big|_0^2 \;d\theta\]
\[ 14000\theta \Big|_0^{2\pi} \le \iint\limits_D f(x,y) \;dA \le 20000\theta \Big|_0^{2\pi}\]
\[ 28000\pi \le \iint\limits_D f(x,y) \;dA \le 40000\pi\]


%-----------------------------QUESTION 5 ---------------------------------%
\newpage \section*{Question 5}



\begin{enumerate}[a)]

\item % a)
Pour répondre à la question, il est plus simple d'inverser les bornes
d'intégration. Soit $D_1:=\{(x,y)\;|\; 0\le x \le 4, \sqrt{x}\le y \le 2\}$, est équivalent à \\
$D_2:=\{(x,y)\;|\; 0\le x \le y^2, 0\le y \le 2\}$ comme en témoigne la figure : 
\begin{figure}[!h]
	\centering
	\includegraphics[scale=0.3]{img5_1.png}
	\captionof{figure}{Zone d'intégration}
\end{figure}
\[J_4= \int_0^4 \int_{\sqrt{x}}^4 \Big[1 + y^2\cos(x\sqrt{y})\Big]\; dydx
=\int_0^2 \int_0^{y^2} \Big[1 + y^2\cos(x\sqrt{y})\Big]\; dxdy =\]
\[ \int_0^2 \Big[x\Big|_0^{y^2} + \frac{y^2}{\sqrt{y}}\sin(x\sqrt{y})\Big|_{x=0}^{x=y^2}\Big] \;dy 
= \int_0^2 \Big[y^2 + y^{3/2}\sin(y^{5/2}) \Big]\; dy = \frac{8}{3} + \int_0^2 y^{3/2}\sin(y^{5/2} \;dy)\]
En posant $u=y^{5/2}$, on calcule $\frac{2}{5}du = y^{3/2}dy$. Par changement de variables : 
\begin{multline}
J_4 = \frac{8}{3} + \int_0^{4\sqrt{2}} \frac{2}{5} \sin(u) \;du = \frac{8}{3} + \frac{-2}{5} \cos(u)\Big|_0^{4\sqrt{2}} \\
= \frac{8}{3} + \frac{-2}{5}\big(\cos(4\sqrt{2}) - \cos(0) \big) = \frac{46}{15} - \frac{2\cos(4\sqrt{2})}{5}
\nonumber
\end{multline}

Pour que l'intégrale représente un volume, il faut que $f(x,y) = 1 + y^2\cos(x\sqrt{y}) \geq 0 \; \forall \; (x,y) \;
\in D_1$. Toutefois, $(\pi, 2) \in D_1$ et $f(\pi, 2) < 0$. Ainsi, par contradiction, l'intégrale ne représente pas un volume. 
\\

\item % b)

Soit $D:=\{(x,y)\;|\; -(2-\frac{y}{2})\le x \le 2-\frac{y}{2}, -4\le y \le 4\}$, il s'agit d'un domaine symétrique
tant pour $x$ que pour $y$ par rapport à 0 : 
\[ J_5 = \iint\limits_D \big[ x^4y + yx^4\sqrt[4]{1-x^4y^4}\; \big] dA =
\int_{-4}^4 \int_{-(2-\frac{y}{2})}^{2-{\frac{y}{2}}} x^4y \;dxdy \;+\; \int_{-(2-\frac{y}{2})}^{2-{\frac{y}{2}}} yx^4\sqrt[4]{1-x^4y^4} \;dxdy\]
Puisque $yx^4\sqrt[4]{1-x^4y^4}$ est impaire en $y$ (soit $y^5$) et qu'on intègre sur un domaine symétrique par rapport à 0,
la seconde partie de l'intégrale est égale à 0 : 
\[ J_5 = \int_{-4}^4 \int_{-(2-\frac{y}{2})}^{2-{\frac{y}{2}}} x^4y \;dxdy \;+ 0 = \int_{-4}^4 y \cdot \frac{x^5}{5}\Big|_{-(2-\frac{y}{2})}^{2-{\frac{y}{2}}}\;dy =
\frac{2}{5}\int_{-4}^4 y(2-\frac{y}{2})^5 \;dy \]. En posant $u = 2-\frac{y}{2}$ et en calculant $du = \frac{-1}{2}dy$ :  
\[J_5 = \frac{8}{5} \int_4^0 (u^6-2 u^5) du = \frac{8}{5} \cdot \frac{u^7}{7}\Big|_4^0 - \frac{16}{5} \cdot \frac{u^6}{6}\big|_4^0 = - \frac{32768}{21} \]
\end{enumerate}


%-----------------------------QUESTION 6 ---------------------------------%
\newpage \section*{Question 6}
Puisque le volume désiré est celui du premier octant, il faut considérer également les plans $x=0$ et $y=0$.
Nous avons les équations $x+z=5 \Leftrightarrow z=5-x$ et $y+2z=4 \Leftrightarrow z=2-\frac{y}{2}$. En posant
$5-x = 2-\frac{y}{2}$, nous pouvons trouver l'intersection des deux plans, soit le plan $y=x-3$. Puisque la
valeur maximale de $z$ est dictée par le plan qui se retrouve sous l'autre plan, il est opportun de procéder 
par deux intégrales, puisqu'une portion du volume est borné par le plan $z=5-x$ alors qu'une seconde est borné 
par le plan $z=2-\frac{y}{2}$. Nous avons ainsi $B_1=\{(x,y,z)\;|\;0\le x\le y+3,\;0\le y\le 4,\; 0 \le z \le 2-\frac{y}{2}\}$. 
Également, $B_2=\{(x,y,z)\;|\;y+3\le x\le 5,\;0\le y\le 4,\; 0 \le z \le 5-x\}$
\[ V = \iiint\limits_{B_1} dV + \iiint\limits_{B_2} dV = \int_0^4\int_0^{y+3}\int_0^{2-\frac{y}{2}} dzdxdy
+ \int_0^4\int_{y+3}^5\int_0^{5-x} dzdxdy =\]
\[ \int_0^4 \left(-\frac{y^2}{2}+\frac{y}{2}+6\right) \, dy + \int_0^4 \left(\frac{y^2}{2}-2y+2\right) \;dy =
 \frac{52}{3} + \frac{8}{3} = 20\]

%-----------------------------QUESTION 7---------------------------------%
\newpage \section*{Question 7}


 \begin{enumerate}[a)]

\item % a)

	\begin{enumerate}[(i)]
	
	\item % (i)
	
	Considérant que la densité dépend de la distance au sommet du triangle qui joint les deux longs côtés, on place celui-ci à l'origine $(0,0)$. Les 
	deux autres sommets du triangle se retrouvent à $(-2,6)$ et $(2,6)$. On peut calculer la distance à l'aide de la formule $\sqrt{x^2+y^2}$.
	Ainsi, $\rho(x,y) = \kappa\sqrt{x^2+y^2}$. Pour calculer le moment par rapport à l'axe des $x$ :
	\[ I_x = \iint\limits_D y^2\rho(x,y) \;dA = \kappa\iint\limits_D y^2 \sqrt{x^2+y^2} \;dA\]
	On pose $D:=\{(r,\theta)\;|\;0\le r\le 6\csc(\theta),\;\arctan(3)\le \theta \le \pi - \arctan(-3) \}$
	\[ I_x = \kappa \int_{\arctan(3)}^{\pi - \arctan(-3)} \int_0^{6\csc(\theta)} r^4\sin^2(\theta) \;drd\theta = 
	\kappa \int_{\arctan(3)}^{\pi - \arctan(-3)} \sin^2(\theta) \frac{r^5}{5}\Big|_0^{6\csc(\theta)} d\theta = \]
	\[ \kappa \frac{7776}{5} \int_{\arctan(3)}^{\pi - \arctan(-3)} \csc^3(\theta)\; d\theta
	= \kappa \frac{15552}{5} \int_{\arctan(3)}^{\frac{\pi}{2}} \csc^3(\theta)\; d\theta\]
	Par la formule de réduction (\#72):
	\[ I_x = \kappa \frac{15552}{5} \Big[\frac{-1}{2} \cdot \csc(\theta)\cot(\theta)  + \frac{1}{2}\ln|\csc(\theta)-\cot(\theta)|
	\Big]_{\arctan(3)}^{\frac{\pi}{2}} \]
	\[= \kappa \;\frac{15552}{5}\Big( \frac{\sqrt{10}}{18} - \frac{1}{2}\ln\big(\frac{\sqrt{10}-1}{3}\big)\Big) \approx 2111.38\kappa\]

	\item % (ii)
	\[ I_y = \iint\limits_D x^2\rho(x,y) \;dA = \kappa\iint\limits_D x^2 \sqrt{x^2+y^2} \;dA \]
	On pose $D:=\{(r,\theta)\;|\;0\le r\le 6\csc(\theta),\;\arctan(3)\le \theta \le \pi - \arctan(-3) \}$
	\[ I_y = \kappa \int_{\arctan(3)}^{\pi - \arctan(-3)} \int_0^{6\csc(\theta)} r^4\cos^2(\theta) \;drd\theta = 
	\kappa \frac{7776}{5} \int_{\arctan(3)}^{\pi - \arctan(-3)} \csc^5(\theta)\cos^2(\theta) \; d\theta \]
	\[ \kappa \frac{15552}{5} \int_{\arctan(3)}^{\frac{\pi}{2}} \frac{1-\sin^2(\theta)}{\sin^5(\theta)} \;d\theta = 
	\kappa \frac{15552}{5} \int_{\arctan(3)}^{\frac{\pi}{2}} \csc^5(\theta)\;d\theta - \int_{\arctan(3)}^{\frac{\pi}{2}} \csc^3(\theta)\; d\theta = \]
	Par la formule de réduction (\#78) et le résultat de l'intégrale précédente : 
	\[ I_y = \kappa \frac{15552}{5} \Big( \frac{-1}{4}\cot(\theta)\csc^3(\theta)\Big|_{\arctan(3)}^{\frac{\pi}{2}} + 
	\frac{3}{4} \int_{\arctan(3)}^{\frac{\pi}{2}} \csc^3(\theta)\; d\theta - \int_{\arctan(3)}^{\frac{\pi}{2}} \csc^3(\theta)\; d\theta \Big) = \]
	\[ \kappa \frac{3888}{5} \Big( \frac{10\sqrt{10}}{81} - \frac{1}{4}\int_{\arctan(3)}^{\frac{\pi}{2}} \csc^3(\theta)\; d\theta\Big) = \]
	\[ \kappa \;\frac{3888}{5} \Big( \frac{10\sqrt{10}}{81} - \frac{1}{4}\big(\frac{\sqrt{10}}{18} - \frac{1}{2}\ln\big(\frac{\sqrt{10}-1}{3} \big)\big)\Big) 
	\approx 237.59\kappa\]
	\\
	\end{enumerate}

\item % b)

 Il est plus facile de tourner la plaque autour de l'axe $A_2$ puisque le moment d'inertie (qui s'oppose à la variation de la vitesse) est plus faible. 

\end{enumerate}

%-----------------------------QUESTION 8---------------------------------%
\newpage \section*{Question 8}
\subsection*{1. Définition de la fonction de densité} 
On énonce que la densité est proportionnelle à la distance au plan : \\ $\{(x,y,z)\;|\;(x,y)\in \mathbb{R},\; z=0\}$. Ainsi, pour tout point 
de la région, la distance au plan $z=0$ est égale à $z$. Ainsi : 
\[ \rho(x,y,z) = \kappa z\]

\subsection*{2. Description de la région E}
En effectuant une projection de la région E sur le plan $D:= xz$, nous avons : 
\[ B:= \big\{(x,y,z)\;|\;(x,z)\in D, -1 \le y \le \frac{z-x^2}{a}\big\} \] Soit : 
\[ B:= \big\{(x,y,z)\;|\; -\sqrt{a} \le x \le \sqrt{a},\; -1 \le y \le \frac{z-x^2}{a},\; 0 \le z \le 2\big\} \]

\subsection*{3. Calcul de la masse}
\[ m = \iiint\limits_E \rho(x,y,z) \; dV = \kappa \int_0^2 \int_{-\sqrt{a}}^{\sqrt{a}} \int_{-1}^{\frac{z-x^2}{a}} z\; dydxdz = 
\kappa \int_0^2 \int_{-\sqrt{a}}^{\sqrt{a}} \big[\frac{z(z-x^2)}{a} + z \big] \;dxdz = \]
\[ \kappa \int_0^2 \Big(\frac{2 z^2}{\sqrt{a}}-\frac{2 \sqrt{a} z}{3}+2 \sqrt{a}\Big) \, dz = \kappa\; \frac{8 (a+2)}{3 \sqrt{a}}\]

\subsection*{4. Calcul des moments}
\[ M_{yx} = \iiint\limits_E x\rho(x,y,z) \; dV = \kappa \int_0^2 \int_{-\sqrt{a}}^{\sqrt{a}} \int_{-1}^{\frac{z-x^2}{a}} xz\; dydxdz = 
\kappa \int_0^2 \int_{-\sqrt{a}}^{\sqrt{a}} xz\big(\frac{z-x^2}{a} + 1\big) \;dxdz\]
Puisque $x$ ici est une fonction impaire intégrée sur un domaine symétrique à 0 : 
\[ M_{yz} = 0\]

\[ M_{xz} = \iiint\limits_E y\rho(x,y,z) \; dV = \kappa \int_0^2 \int_{-\sqrt{a}}^{\sqrt{a}} \int_{-1}^{\frac{z-x^2}{a}} yz\; dydxdz = 
\kappa \int_0^2 \int_{-\sqrt{a}}^{\sqrt{a}} \big(\frac{z\left(z-x^2\right)^2}{2a^2}-\frac{z}{2}\big)\;dxdz = \]
\[\kappa \int_0^2 \frac{15z^3-10a^2-12a^2z}{15a^{\frac{3}{2}}}\;dz = 
\kappa\; \frac{4\left(-18a^2-20a+45\right)}{45a^{\frac{3}{2}}}\] \\

\[ M_{xy} = \iiint\limits_E z\rho(x,y,z) \; dV = \kappa \int_0^2 \int_{-\sqrt{a}}^{\sqrt{a}} \int_{-1}^{\frac{z-x^2}{a}} z^2\; dydxdz = 
\kappa \int_0^2 \int_{-\sqrt{a}}^{\sqrt{a}} \frac{z^2\left(z-x^2\right)}{a}+z^2 \; dxdz = \]
\[ \kappa \int_0^2 \frac{2z^3}{\sqrt{a}} + \frac{4\sqrt{a}z^2}{3} \;dz = 
\kappa \; \Big[\frac{8}{\sqrt{a}}+\frac{32\sqrt{a}}{9}\Big]  \]

\subsection*{5. Coordonnées en x et y du centre de masse}
\[ \overline{x} = \frac{M_{yz}}{m} = \frac{0}{\frac{8 (a+2)}{3 \sqrt{a}}} = 0 \]
\[ \overline{y} = \frac{M_{xz}}{m} \frac{\frac{4\left(-18a^2-20a+45\right)}{45a^{\frac{3}{2}}}}{\frac{8 (a+2)}{3 \sqrt{a}}}
= (-18 a^2-20 a+45)/(30 a (a+2)) \]

\subsection*{6. Coordonnée en z du centre de masse}
\[ \overline{z} = \frac{M_{xy}}{m} = \frac{\frac{8}{\sqrt{a}}+\frac{32\sqrt{a}}{9}}{\frac{8 (a+2)}{3 \sqrt{a}}} = 
\frac{3 \left(\frac{32 \sqrt{a}}{9}+\frac{8}{\sqrt{a}}\right) \sqrt{a}}{8 (a+2)}\]

\subsection*{7. Inégalité correspondant à la question}
\[\frac{3 \left(\frac{32 \sqrt{a}}{9}+\frac{8}{\sqrt{a}}\right) \sqrt{a}}{8 (a+2)} > 0\] 

\subsection*{8. Solution de l'inégalité}
\[\frac{3 \left(\frac{32 \sqrt{a}}{9}+\frac{8}{\sqrt{a}}\right) \sqrt{a}}{8 (a+2)} > 0 \Leftrightarrow a > 0 \; \text{dans} \;\mathbb{R} \]

\subsection*{9. Conclusion}
Le centre de masse du solide est situé au dessus de sa base lorsque $a>0$.

\end{document}
