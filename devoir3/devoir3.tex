%--------------------------INITIALISATION DU DOCUMENT---------------------%
%											%														%											%
%                        >>>> Ne pas modifier cette partie <<<<			%
																																					
\documentclass[letterpaper,12pt,oneside,final]{book}



%%
%%  Version: 2014-10-28
%%
%%  Accepte les caractères accentués dans le document (UTF-8).
\usepackage[utf8]{inputenc}
%%
%% Support pour l'anglais et le français (français par défaut).
%\usepackage[cyr]{aeguill}
\usepackage{lmodern}      % Police de caractères plus complète et généralement indistinguable visuellement de la police standard de LaTeX (Computer Modern).
\usepackage[T1]{fontenc}  % Bon encodage des caractères pour qu'Acrobat Reader reconnaisse les accents et les ligatures telles que ffi.
\usepackage[english,frenchb]{babel} % le langage par défaut est le dernier de la liste, c'est-à-dire français
%%
%% Charge le module d'affichage graphique.
\usepackage{graphicx}
\usepackage{epstopdf}  % Permet d'utiliser des .eps avec pdfLaTeX.
%%
%% Recherche des images dans les répertoires.
\graphicspath{{./images/}{./dia/}{./gnuplot/}}
%%
%% Un float peut apparaître seulement après sa définition, jamais avant.
\usepackage{flafter,placeins}
%%
%% Utilisation de natbib pour les citations et la bibliographie.
\usepackage{natbib}
%%
%% Autres packages.
\usepackage{amsmath,color,soulutf8,longtable,colortbl,setspace,ifthen,xspace,url,pdflscape,tikz,pgfplots}
%%
%% Support des acronymes.
\usepackage[nolist]{acronym}
\onehalfspacing                % Interligne 1.5.
%%
%% Définition d'un style de page avec seulement le numéro de page à
%% droite. On s'assure aussi que le style de page par défaut soit
%% d'afficher le numéro de page en haut à droite.
\usepackage{fancyhdr}
\fancypagestyle{pagenumber}{\fancyhf{}\fancyhead[R]{\thepage}}
\renewcommand\headrulewidth{0pt}
\makeatletter
\let\ps@plain=\ps@pagenumber
\makeatother
%%
%% Module qui permet la création des bookmarks dans un fichier PDF.
%\usepackage[dvipdfm]{hyperref}
\usepackage{hyperref}
\usepackage{caption}  % Hyperlien vers la figure plutôt que son titre.

\usepackage{esint}
\usepackage{geometry}
\usepackage{enumerate}



\setlength{\parindent}{0pt}
\newcommand{\norme}[1]{\left\Vert #1\right\Vert}
\usepackage{amssymb}
\begin{document}
%--------------------------------------------------------------------------------------%

%--------------------------PAGE DE COUVERTURE------------------------------%

% A REMPLIR PAR L'ETUDIANT: 

\newcommand\monPrenom{William}		%PRENOM
\newcommand\monNom{Trépanier}			%NOM
\newcommand\monMatricule{1952594}	%MATRICULE
\newcommand\monGroupe{1}		%GROUPE

%------------------------ Ne pas modifier la ligne suivante --------------%
%\newgeometry{tmargin=2.0cm, bmargin=2.0cm, lmargin=2.25cm, rmargin=2.25cm, headsep=1.0cm}
\newgeometry{top=2cm}
\definecolor{gris1}{gray}{0.75}

\newcommand{\encadre}[1]{
\setlength\fboxsep{5mm}\setlength\fboxrule{1pt}
\begin{center}
\fcolorbox{black}{gris1}{
\begin{minipage}{0.94\textwidth}{#1}\end{minipage}}
\end{center}}

% encadre blanc
\newcommand{\boite}[1]{
\setlength\fboxsep{5mm}\setlength\fboxrule{1pt}
\begin{center}
\fcolorbox{black}{white}{
\begin{minipage}{0.5\textwidth}{#1}\end{minipage}}
\end{center}}


%\begin{document}

\thispagestyle{empty}

{
\centering

\encadre{
\begin{center}
\bf
{\Large \scshape 
Polytechnique Montr\'eal
\\
D\'epartement de Math\'ematiques et de G\'enie Industriel
}
\\
{\Huge
\

MTH1102 - Calcul II
\\
\'Et\'e 2019 - Trimestre court

\

Devoir 4

}
\end{center}
}

\vfill

\fcolorbox{black}{white}{
\begin{minipage}{0.94\linewidth}

\vspace{5mm}

{\bf \Large Nom: }\monNom \hspace{20mm} {\bf \Large Pr\'enom: }\monPrenom%\rule[-1mm]{56mm}{0.6pt}

\vspace{8mm}

{\bf \Large Matricule: }\monMatricule \hspace{20mm} {\bf \Large Section: }\monGroupe

%\vspace{8mm}
%
%{\bf \Large Signature: }\rule[-1mm]{126mm}{0.6pt}
%
%
%\vspace{5mm}

\end{minipage}}

\vfill


{
\renewcommand{\arraystretch}{1.5}
\begin{center}
\begin{tabular}{|c|c|c||c|} \hline
{\bf \Large Question}							& {\bf \Large Autres}			& 	{\bf \Large Bonus}	&  \\ 
{\bf \Large corrig\'ee}						& {\bf \Large  questions}	&		{\bf \Large \LaTeX}	& {\bf \Large Total} \\ \hline
\hspace{20mm}			{\Huge \strut}	& \hspace{20mm}						&		\hspace{20mm}				&\hspace{20mm} \\
\hspace{20mm}			{\Huge \strut}	& \hspace{20mm}						&		\hspace{20mm} 			& \hspace{20mm} {\Large /10} \\ \hline
\end{tabular}
\end{center}
}


\vfill

}

\restoregeometry
%\end{document}
%-------------------------------------------------------------------------%


%========================= Début des réponses ============================%


%-----------------------------QUESTION 1----------------------------------%
\section*{Question 1}

 % A REMPLIR PAR L'ETUDIANT:


%-----------------------------QUESTION 2---------------------------------%
\newpage \section*{Question 2}
Il faut déterminer la courbe d'intersection :
\begin{align*}
	x = 2 \Rightarrow z = 4 + y^2
\end{align*}
En posant $t=y$:
\[\begin{cases}
	x=2\\
	y=t\\
	z=4+t^2\\
\end{cases}
t \in \mathbb{R}\]

Pour les deux points :
\begin{gather*}
	(2,-1,5) \Leftrightarrow t=-1 \\
	(2,1,5) \Leftrightarrow t=1
\end{gather*}

Ainsi:
\begin{gather*}
	L = \int_{a}^{b} \sqrt{\Big(\frac{dx}{dt}\Big)^2+
		\Big(\frac{dy}{dt}\Big)^2 + \Big(\frac{dz}{dt}\Big)^2} \,dt
		= \int_{-1}^{1} \sqrt{\big(0\big)^2+
		\big(1\big)^2 + \big(2t\big)^2} \,dt \\
	= \int_{-1}^{1} \sqrt{\big(1\big)^2 + \big(2t\big)^2} \,dt
\end{gather*}

En posant $u=2t$ et en calculant $du = \frac{1}{2}dt$ :
\begin{gather*}
	L = \frac{1}{2}\int_{-2}^{2} \sqrt{1^2 + u^2} \,du
\end{gather*}

En utilisant la formule de réduction \#21:
\begin{gather*}
	L = \frac{1}{2}\Big[\frac{u}{2}\sqrt{1^2 + u^2} + 
		\frac{1}{2}\ln(u+\sqrt{1^2 + u^2})\Big]_{u=-2}^{u=2} \\
	= \frac{1}{2} \Big(\sqrt{5} + \frac{1}{2}\big(\ln(2+\sqrt{5})\big)
		-\Big(-\sqrt{5}+\frac{1}{2}\big(\ln(\sqrt{5}-2)\big)\Big)\Big) \\ 
	= \frac{1}{2}\Big(2\sqrt{5}+\frac{1}{2}\ln\big(\frac{2+\sqrt{5}}
		{\sqrt{5}-2}\big) \Big) = \sqrt{5} + 
		\frac{1}{4}\ln\big( \frac{2+\sqrt{5}}{\sqrt{5}-2}\big) \\
	\approx 2.9579
\end{gather*}

%-----------------------------QUESTION 3---------------------------------%
\newpage \section*{Question 3}


\begin{enumerate}[a)]

\item % a)

Pour trouver la vitesse au point (1,3):
\begin{gather*}
	\vec{F}(1,3) = (1\cdot3 -2)\vec{i} + (3^2-10)\vec{j} = \vec{i} - \vec{j}
\end{gather*}

Puisque la vitesse correspond à un déplacement par unité de temps :
\begin{gather*}
	\Delta t = 1.05-1 = 0.05
\end{gather*}

On estime ainsi le déplacement en multipliant la vitesse par une unité
	de temps :
\begin{gather*}
	\Delta x = 1 \cdot 0.05 = 0.05 \\
	\Delta y = -1 \cdot 0.05 = -0.05
\end{gather*}

Pour trouver la position :
\begin{gather*}
	x = 1 + 0.05 = 1.05 \\
	y = 3 - 0.05 = 2.95
\end{gather*}

La position approximative à l'instant $t=1.05$ est (1.05, 2.95). \\

\item % b)

Puisque les vecteurs du champ sont parallèles aux lignes de courant :
\begin{gather*}
	\vec{r}\;'(t) = \vec{G}(x(t),y(t)) \Rightarrow \\
	x'(t)\vec{i} + y'(t)\vec{j} = \vec{i} + (1+y^2)\vec{j}
\end{gather*}

Nous avons ainsi :
\begin{gather*}
	\frac{dx}{dt} = 1 \Rightarrow dx = dt\\
	\frac{dy}{dt} = 1 + y^2 \Rightarrow \frac{dy}{1+y^2} = dt \\
\end{gather*}

On peut isoler x et y en fonction de t en intégrant des deux côtés:
\begin{gather*}
	\int dx = \int dt \Rightarrow x = t + C_1 \\
	\int \frac{1}{1^2 + y^2} \,dy = \int dt \Rightarrow 
		\arctan y = t + C_2 \Rightarrow y = \tan(t+C_2) \\
\end{gather*}

On pose $t=0$ au point (1,1) :
\begin{gather*}
	\vec{r}\;(0) = C_1\vec{i} + \tan(C_2)\vec{j} = \vec{i} + \vec{j} \\
\end{gather*}

On choisit ainsi :
\begin{gather*}
	C_1 = 1 \\
	C_2 = \frac{\pi}{4} \\
	\vec{r}(t) = (t+1)\vec{i} + \tan(t+\frac{\pi}{4})
\end{gather*}

Nous avons ainsi la paramétrisation suivante :
\[\begin{cases}
	x = t + 1 \\
	y = \tan(t+\frac{\pi}{4})
\end{cases}
t \in \mathbb{R} \backslash \big\{\frac{\pi}{4} + k\pi,\, k\in\mathbb{Z}\big\} \]
\end{enumerate}



%-----------------------------QUESTION 4---------------------------------%
\newpage \section*{Question 4}



 \begin{enumerate}[a)]

\item % a)

Il faut d'abord paramétrer $C$:
\begin{gather*}
	\vec{r}\;'(t) = (1-t)(\vec{i}-\vec{j}) + t(\vec{i}+2\vec{j}+3\vec{k}) =
		\vec{i}-\vec{j}+3t\vec{j}+3t\vec{k}	
\end{gather*}
Nous avons la paramétrisation suivante :
\[ \begin{cases}
	x = 1\\
	y = 3t - 1\\
	z = 3t
\end{cases}
t \in [0,1]\]

Ainsi :
\begin{gather*}
	J_1 = \int\limits_C (xy + z)\,ds = \int_0^1 \big(1\cdot 
		(3t-1)+3t\big)\sqrt{(0)^2 + (3)^2 + (3)^2}\,dt \\
	= 3\sqrt{2}\int_0^1 (6t-1)\,dt = 3\sqrt{2}\big[3t^2-t\big]_{t=0}^{t=1}
		=6\sqrt{2}
\end{gather*}

\item % b)

En utilisant la même paramétrisation qu'en a), nous avons :
\begin{gather*}
	\vec{r}(t) = \vec{i} + (3t-1)\vec{j} + 3t\vec{k} \\
	\vec{r}\;'(t) = 3\vec{j} + 3\vec{k} \\
	\vec{F}(\vec{r}(t)) = 3t\vec{i} + \vec{j} + (3t-1)^2\vec{k}
\end{gather*}

Quant à l'intégrale :
\begin{gather*}
	J_2 = \int\limits_C \vec{F}\cdot\vec{dr} = \int_0^1 \vec{F}(\vec{r}(t))
		\cdot \vec{r}\;'(t)\,dt = 
		\int_0^1 \big(3t\vec{i} + \vec{j} + (3t-1)^2\vec{k}\big)\cdot\big(
		3\vec{j} + 3\vec{k}\big)\;dt \\
	= \int_0^1 (3+3(3t-1)^2)\,dt = \int_0^1 (27t^2-18t+6)\,dt = 
		\Big[\frac{27t^3}{3} - \frac{18t^2}{2} + 6t\Big]_{t=0}^{t=1}
		= 6
\end{gather*}

\end{enumerate}


%-----------------------------QUESTION 5 ---------------------------------%
\newpage \section*{Question 5}



 \begin{enumerate}[a)]

\item % a)

 % A REMPLIR PAR L'ETUDIANT:

\item % b)

 % A REMPLIR PAR L'ETUDIANT:

\item % c)

 % A REMPLIR PAR L'ETUDIANT:

\end{enumerate}


%-----------------------------QUESTION 6 ---------------------------------%
\newpage \section*{Question 6}



 % A REMPLIR PAR L'ETUDIANT:


%-----------------------------QUESTION 7---------------------------------%
\newpage \section*{Question 7}



\begin{enumerate}[a)]

\item % a)

 % A REMPLIR PAR L'ETUDIANT:

\item % b)

 % A REMPLIR PAR L'ETUDIANT:

\item % c)

	\begin{enumerate}[(i)]

	\item % (i)

	% A REMPLIR PAR L'ETUDIANT:
	
	\item % (ii)

	% A REMPLIR PAR L'ETUDIANT:
	
	\end{enumerate}

\end{enumerate}





\end{document}
