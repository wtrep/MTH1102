%--------------------------INITIALISATION DU DOCUMENT---------------------%
%											%														%											%
%                        >>>> Ne pas modifier cette partie <<<<			%
																																					
\documentclass[letterpaper,12pt,oneside,final]{book}



%%
%%  Version: 2014-10-28
%%
%%  Accepte les caractères accentués dans le document (UTF-8).
\usepackage[utf8]{inputenc}
%%
%% Support pour l'anglais et le français (français par défaut).
%\usepackage[cyr]{aeguill}
\usepackage{lmodern}      % Police de caractères plus complète et généralement indistinguable visuellement de la police standard de LaTeX (Computer Modern).
\usepackage[T1]{fontenc}  % Bon encodage des caractères pour qu'Acrobat Reader reconnaisse les accents et les ligatures telles que ffi.
\usepackage[english,frenchb]{babel} % le langage par défaut est le dernier de la liste, c'est-à-dire français
%%
%% Charge le module d'affichage graphique.
\usepackage{graphicx}
\usepackage{epstopdf}  % Permet d'utiliser des .eps avec pdfLaTeX.
%%
%% Recherche des images dans les répertoires.
\graphicspath{{./images/}{./dia/}{./gnuplot/}}
%%
%% Un float peut apparaître seulement après sa définition, jamais avant.
\usepackage{flafter,placeins}
%%
%% Utilisation de natbib pour les citations et la bibliographie.
\usepackage{natbib}
%%
%% Autres packages.
\usepackage{amsmath,color,soulutf8,longtable,colortbl,setspace,ifthen,xspace,url,pdflscape,tikz,pgfplots}
%%
%% Support des acronymes.
\usepackage[nolist]{acronym}
\onehalfspacing                % Interligne 1.5.
%%
%% Définition d'un style de page avec seulement le numéro de page à
%% droite. On s'assure aussi que le style de page par défaut soit
%% d'afficher le numéro de page en haut à droite.
\usepackage{fancyhdr}
\fancypagestyle{pagenumber}{\fancyhf{}\fancyhead[R]{\thepage}}
\renewcommand\headrulewidth{0pt}
\makeatletter
\let\ps@plain=\ps@pagenumber
\makeatother
%%
%% Module qui permet la création des bookmarks dans un fichier PDF.
%\usepackage[dvipdfm]{hyperref}
\usepackage{hyperref}
\usepackage{caption}  % Hyperlien vers la figure plutôt que son titre.

\usepackage{esint}
\usepackage{geometry}
\usepackage{enumerate}



\setlength{\parindent}{0pt}
\newcommand{\norme}[1]{\left\Vert #1\right\Vert}
\usepackage{amssymb}
\begin{document}
%--------------------------------------------------------------------------------------%

%--------------------------PAGE DE COUVERTURE------------------------------%

% A REMPLIR PAR L'ETUDIANT: 

\newcommand\monPrenom{William}		%PRENOM
\newcommand\monNom{Trépanier}			%NOM
\newcommand\monMatricule{1952594}	%MATRICULE
\newcommand\monGroupe{1}		%GROUPE

%------------------------ Ne pas modifier la ligne suivante --------------%
%\newgeometry{tmargin=2.0cm, bmargin=2.0cm, lmargin=2.25cm, rmargin=2.25cm, headsep=1.0cm}
\newgeometry{top=2cm}
\definecolor{gris1}{gray}{0.75}

\newcommand{\encadre}[1]{
\setlength\fboxsep{5mm}\setlength\fboxrule{1pt}
\begin{center}
\fcolorbox{black}{gris1}{
\begin{minipage}{0.94\textwidth}{#1}\end{minipage}}
\end{center}}

% encadre blanc
\newcommand{\boite}[1]{
\setlength\fboxsep{5mm}\setlength\fboxrule{1pt}
\begin{center}
\fcolorbox{black}{white}{
\begin{minipage}{0.5\textwidth}{#1}\end{minipage}}
\end{center}}


%\begin{document}

\thispagestyle{empty}

{
\centering

\encadre{
\begin{center}
\bf
{\Large \scshape 
Polytechnique Montr\'eal
\\
D\'epartement de Math\'ematiques et de G\'enie Industriel
}
\\
{\Huge
\

MTH1102 - Calcul II
\\
\'Et\'e 2019 - Trimestre court

\

Devoir 4

}
\end{center}
}

\vfill

\fcolorbox{black}{white}{
\begin{minipage}{0.94\linewidth}

\vspace{5mm}

{\bf \Large Nom: }\monNom \hspace{20mm} {\bf \Large Pr\'enom: }\monPrenom%\rule[-1mm]{56mm}{0.6pt}

\vspace{8mm}

{\bf \Large Matricule: }\monMatricule \hspace{20mm} {\bf \Large Section: }\monGroupe

%\vspace{8mm}
%
%{\bf \Large Signature: }\rule[-1mm]{126mm}{0.6pt}
%
%
%\vspace{5mm}

\end{minipage}}

\vfill


{
\renewcommand{\arraystretch}{1.5}
\begin{center}
\begin{tabular}{|c|c|c||c|} \hline
{\bf \Large Question}							& {\bf \Large Autres}			& 	{\bf \Large Bonus}	&  \\ 
{\bf \Large corrig\'ee}						& {\bf \Large  questions}	&		{\bf \Large \LaTeX}	& {\bf \Large Total} \\ \hline
\hspace{20mm}			{\Huge \strut}	& \hspace{20mm}						&		\hspace{20mm}				&\hspace{20mm} \\
\hspace{20mm}			{\Huge \strut}	& \hspace{20mm}						&		\hspace{20mm} 			& \hspace{20mm} {\Large /10} \\ \hline
\end{tabular}
\end{center}
}


\vfill

}

\restoregeometry
%\end{document}
%-------------------------------------------------------------------------%


%========================= Début des réponses ============================%


%-----------------------------QUESTION 1----------------------------------%
\section*{Question 1}

 % A REMPLIR PAR L'ETUDIANT:


%-----------------------------QUESTION 2---------------------------------%
\newpage \section*{Question 2}
Il faut déterminer la courbe d'intersection :
\begin{align*}
	x = 2 \Rightarrow z = 4 + y^2
\end{align*}
En posant $t=y$:
\[\begin{cases}
	x=2\\
	y=t\\
	z=4+t^2\\
\end{cases}
t \in \mathbb{R}\]

Pour les deux points :
\begin{gather*}
	(2,-1,5) \Leftrightarrow t=-1 \\
	(2,1,5) \Leftrightarrow t=1
\end{gather*}

Ainsi:
\begin{gather*}
	L = \int_{a}^{b} \sqrt{\Big(\frac{dx}{dt}\Big)^2+
		\Big(\frac{dy}{dt}\Big)^2 + \Big(\frac{dz}{dt}\Big)^2} \,dt
		= \int_{-1}^{1} \sqrt{\big(0\big)^2+
		\big(1\big)^2 + \big(2t\big)^2} \,dt \\
	= \int_{-1}^{1} \sqrt{\big(1\big)^2 + \big(2t\big)^2} \,dt
\end{gather*}

En posant $u=2t$ et en calculant $du = \frac{1}{2}dt$ :
\begin{gather*}
	L = \frac{1}{2}\int_{-2}^{2} \sqrt{1^2 + u^2} \,du
\end{gather*}

En utilisant la formule de réduction \#21:
\begin{gather*}
	L = \frac{1}{2}\Big[\frac{u}{2}\sqrt{1^2 + u^2} + 
		\frac{1}{2}\ln(u+\sqrt{1^2 + u^2})\Big]_{u=-2}^{u=2} \\
	= \frac{1}{2} \Big(\sqrt{5} + \frac{1}{2}\big(\ln(2+\sqrt{5})\big)
		-\Big(-\sqrt{5}+\frac{1}{2}\big(\ln(\sqrt{5}-2)\big)\Big)\Big) \\ 
	= \frac{1}{2}\Big(2\sqrt{5}+\frac{1}{2}\ln\big(\frac{2+\sqrt{5}}
		{\sqrt{5}-2}\big) \Big) = \sqrt{5} + 
		\frac{1}{4}\ln\big( \frac{2+\sqrt{5}}{\sqrt{5}-2}\big) \\
	\approx 2.9579
\end{gather*}

%-----------------------------QUESTION 3---------------------------------%
\newpage \section*{Question 3}


\begin{enumerate}[a)]

\item % a)

Pour trouver la vitesse au point (1,3):
\begin{gather*}
	\vec{F}(1,3) = (1\cdot3 -2)\vec{i} + (3^2-10)\vec{j} = \vec{i} - \vec{j}
\end{gather*}

Puisque la vitesse correspond à un déplacement par unité de temps :
\begin{gather*}
	\Delta t = 1.05-1 = 0.05
\end{gather*}

On estime ainsi le déplacement en multipliant la vitesse par une unité
	de temps :
\begin{gather*}
	\Delta x = 1 \cdot 0.05 = 0.05 \\
	\Delta y = -1 \cdot 0.05 = -0.05
\end{gather*}

Pour trouver la position :
\begin{gather*}
	x = 1 + 0.05 = 1.05 \\
	y = 3 - 0.05 = 2.95
\end{gather*}

La position approximative à l'instant $t=1.05$ est (1.05, 2.95). \\

\item % b)

Puisque les vecteurs du champ sont parallèles aux lignes de courant :
\begin{gather*}
	\vec{r}\;'(t) = \vec{G}(x(t),y(t)) \Rightarrow \\
	x'(t)\vec{i} + y'(t)\vec{j} = \vec{i} + (1+y^2)\vec{j}
\end{gather*}

Nous avons ainsi :
\begin{gather*}
	\frac{dx}{dt} = 1 \Rightarrow dx = dt\\
	\frac{dy}{dt} = 1 + y^2 \Rightarrow \frac{dy}{1+y^2} = dt \\
\end{gather*}

On peut isoler x et y en fonction de t en intégrant des deux côtés:
\begin{gather*}
	\int dx = \int dt \Rightarrow x = t + C_1 \\
	\int \frac{1}{1^2 + y^2} \,dy = \int dt \Rightarrow 
		\arctan y = t + C_2 \Rightarrow y = \tan(t+C_2) \\
\end{gather*}

On pose $t=0$ au point (1,1) :
\begin{gather*}
	\vec{r}\;(0) = C_1\vec{i} + \tan(C_2)\vec{j} = \vec{i} + \vec{j} \\
\end{gather*}

On choisit ainsi :
\begin{gather*}
	C_1 = 1 \\
	C_2 = \frac{\pi}{4} \\
	\vec{r}(t) = (t+1)\vec{i} + \tan(t+\frac{\pi}{4})
\end{gather*}

Nous avons ainsi la paramétrisation suivante :
\[\begin{cases}
	x = t + 1 \\
	y = \tan(t+\frac{\pi}{4})
\end{cases}
t \in \mathbb{R} \backslash \big\{\frac{\pi}{4} + k\pi,\, k\in\mathbb{Z}\big\} \]
\end{enumerate}



%-----------------------------QUESTION 4---------------------------------%
\newpage \section*{Question 4}



 \begin{enumerate}[a)]

\item % a)

Il faut d'abord paramétrer $C$:
\begin{gather*}
	\vec{r}\;'(t) = (1-t)(\vec{i}-\vec{j}) + t(\vec{i}+2\vec{j}+3\vec{k}) =
		\vec{i}-\vec{j}+3t\vec{j}+3t\vec{k}	
\end{gather*}
Nous avons la paramétrisation suivante :
\[ \begin{cases}
	x = 1\\
	y = 3t - 1\\
	z = 3t
\end{cases}
t \in [0,1]\]

Ainsi :
\begin{gather*}
	J_1 = \int\limits_C (xy + z)\,ds = \int_0^1 \big(1\cdot 
		(3t-1)+3t\big)\sqrt{(0)^2 + (3)^2 + (3)^2}\,dt \\
	= 3\sqrt{2}\int_0^1 (6t-1)\,dt = 3\sqrt{2}\big[3t^2-t\big]_{t=0}^{t=1}
		=6\sqrt{2}
\end{gather*}

\item % b)

En utilisant la même paramétrisation qu'en a), nous avons :
\begin{gather*}
	\vec{r}(t) = \vec{i} + (3t-1)\vec{j} + 3t\vec{k} \\
	\vec{r}\;'(t) = 3\vec{j} + 3\vec{k} \\
	\vec{F}(\vec{r}(t)) = 3t\vec{i} + \vec{j} + (3t-1)^2\vec{k}
\end{gather*}

Quant à l'intégrale :
\begin{gather*}
	J_2 = \int\limits_C \vec{F}\cdot\vec{dr} = \int_0^1 \vec{F}(\vec{r}(t))
		\cdot \vec{r}\;'(t)\,dt = 
		\int_0^1 \big(3t\vec{i} + \vec{j} + (3t-1)^2\vec{k}\big)\cdot\big(
		3\vec{j} + 3\vec{k}\big)\;dt \\
	= \int_0^1 (3+3(3t-1)^2)\,dt = \int_0^1 (27t^2-18t+6)\,dt = 
		\Big[\frac{27t^3}{3} - \frac{18t^2}{2} + 6t\Big]_{t=0}^{t=1}
		= 6
\end{gather*}

\end{enumerate}


%-----------------------------QUESTION 5 ---------------------------------%
\newpage \section*{Question 5}



 \begin{enumerate}[a)]

\item % a)

On cherche $f$ tel que $\vec{F} = \nabla f$:
\begin{gather*}
	f_x = yze^{xy} - 2xy \Rightarrow f = ze^{xy} -x^2y \\
	f_y = xze^{xy} - x^2 \Rightarrow f = ze^{xy} -x^2y \\
	f_z = e^{xy} + 1 \Rightarrow f = ze^{xy} + z
\end{gather*}
	
Ainsi :
\begin{gather*}
	f = ze^{xy} -x^2y + z
\end{gather*}

\item % b)

Soit l'arc de cercle C. Selon le théorème fondamental des intégrales curvilignes 
	et considérantque le travail est une force conservative :
\begin{gather*}
	W = \int\limits_C \vec{F}\cdot d\vec{r} = \int\limits_C \nabla 
		f\cdot d\vec{r} = f(0,2,1) - f(1,2,3) \\
	= (1\cdot e^0 - 0 + 1) - (3\cdot e^2 - 2 + 3) = 2-3e^2-1 = 1-3e^2
\end{gather*}

\item % c)

Soit $C$, le cercle situé  dans le plan $z=1-x-y$ et $\{\vec{r}(a), 
	\vec{r}(b)\,|\,\vec{r}(a) = \vec{r}(b)\} \in C$. Selon le théorème 
	fondamental des intégrales curvilignes et considérant que le travail 
	est une force conservative :

\begin{gather*}
	W = \int\limits_C \vec{F}\cdot d\vec{r} = \int\limits_C \nabla 
		f\cdot d\vec{r} = f(b) - f(a) = f(b) - f(b) = 0
\end{gather*}

\end{enumerate}


%-----------------------------QUESTION 6 ---------------------------------%
\newpage \section*{Question 6}

Soit $C$ la courbe paramétrée par la fonction vectorielle $\vec{r}(t)$ où
	$a\le t \le b$. Posons $\vec{r}(b) = p\vec{i}+q\vec{j}+r\vec{k}$ et
	$\vec{r}(a) = l\vec{i}+m\vec{j}+n\vec{k}$ :

\begin{gather*}
	\int\limits_C \vec{x}\cdot d\vec{r} = \int\limits_C x\,dx+y\,dy+z\,dz \\
	= \int_l^p x\,dx + \int_m^q y\,dy + \int_n^r z\,dz \\
	= \frac{x^2}{2}\Big|_l^p+\frac{y^2}{2}\Big|_m^q+\frac{z^2}{2}\Big|_n^r\\
	= \frac{1}{2}\,\Big[p^2-l^2+q^2-m^2+r^2-n^2 \Big] \\
	= \frac{1}{2}\,\Big[(p^2+q^2+r^2)-(l^2+m^2+n^2)\Big] \\
	= \frac{1}{2}\,\Big[\norme{\vec{r}(b)}^2-\norme{\vec{r}(a)}^2\Big]
\end{gather*}


%-----------------------------QUESTION 7---------------------------------%
\newpage \section*{Question 7}



\begin{enumerate}[a)]

\item % a)

L'intersection entre les deux boucles correspond au point où 
	$\cos(t)-\sin(t)=0$. Ainsi :

\begin{gather*}
	\cos(t) - \sin(t) = 0 \Leftrightarrow \cos(t) = \sin(t) 
		\Rightarrow t = \big\{\frac{\pi}{4}, \frac{5\pi}{4}\big\}
		\;t\in [0,2\pi]
\end{gather*}

Pour obtenir la boucle de gauche, on pose $a = \frac{\pi}{4}$ et 
	$c=\frac{5\pi}{4}$. Pour la boucle de droite, $b=c+\pi=\frac{9\pi}{4}$.
	Ainsi, l'intervalle est $\big[\frac{\pi}{4},\frac{9\pi}{4}\big]$ dont
	le point milieu est $\frac{5\pi}{4}$. \\

\item % b)

Pour calculer l'aire en utilisant le théorème de Green, il faut utiliser deux
	intégrales, car la boucle de droite est d'orientation négative alors que la
	boucle de droite est orientée positivement :

\begin{gather*}
	A = -\oint\limits_C y\,dx = -\oint(\cos(t)\sin(t)+\sin^2(t))
		(-\sin(t)\cos(t))\,dt \\
	= \int_{\frac{\pi}{4}}^{\frac{5\pi}{4}} (\sin^3(t) + 2\sin^2(t)\cos(t) 
		+ \sin(t)\cos^2(t))\,dt - \\
	\int_{\frac{5\pi}{4}}^{\frac{9\pi}{4}} (\sin^3(t) + 2\sin^2(t)\cos(t) 
		+ \sin(t)\cos^2(t))\,dt \\
	= I_1 - I_2 
\end{gather*}
\begin{gather*}
	I_1 = \int_{\frac{\pi}{4}}^{\frac{5\pi}{4}} (\sin^3(t) + 2\sin^2(t)\cos(t) 
	+ \sin(t)\cos^2(t))\,dt \\
	= \int_{\frac{\pi}{4}}^{\frac{5\pi}{4}} (\sin(t)(1-\cos^2(t)) + 
		2\sin^2(t)\cos(t) + \sin(t)\cos^2(t))\,dt \\
	= \int_{\frac{\pi}{4}}^{\frac{5\pi}{4}} (\sin(t) + 
	2\sin^2(t)\cos(t))\,dt \\
	\text{En posant } u=\sin(t) \text{ et }du = \cos(t)dt\\ 
	= -\cos(t)\Big|_{\frac{\pi}{4}}^{\frac{5\pi}{4}} +
		2\int_{\frac{\sqrt{2}}{2}}^{-\frac{\sqrt{2}}{2}} u^2\,\\
	= \sqrt{2} - \frac{2}{3}\cdot u^3\Big|_{-\frac{\sqrt{2}}{2}}^
		{\frac{\sqrt{2}}{2}} \\
	= \sqrt{2} - \frac{\sqrt{2}}{3} = \frac{2\sqrt{2}}{3}
\end{gather*}
	
\begin{gather*}
	I_2 = \int_{\frac{5\pi}{4}}^{\frac{9\pi}{4}}(\sin^3(t) + 2\sin^2(t)
		\cos(t) + \sin(t)\cos^2(t))\,dt \\
\end{gather*}
En appliquant la même démarche, mais avec les bornes particulières de
	$I_2$, on obtient:
\begin{gather*}
	I_2 = -\frac{2\sqrt{2}}{3} = -I_1
\end{gather*}

Ainsi, pour l'aire:
\begin{gather*}
	A = I_1 - I_2 = I_1 + I_1 = 2\cdot \frac{2\sqrt{2}}{3} = 
		\frac{4\sqrt{2}}{3}
\end{gather*}

\item % c)

	\begin{enumerate}[(i)]

	\item % (i)

	% A REMPLIR PAR L'ETUDIANT:
	
	\item % (ii)

	% A REMPLIR PAR L'ETUDIANT:
	
	\end{enumerate}

\end{enumerate}





\end{document}
